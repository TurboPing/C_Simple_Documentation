\hypertarget{mainloop__glib_8c}{
\section{mainloop\_\-glib.c File Reference}
\label{mainloop__glib_8c}\index{mainloop_glib.c@{mainloop\_\-glib.c}}
}
Example for main loop using glib 2.  


{\tt \#include $<$unistd.h$>$}\par
{\tt \#include $<$stdlib.h$>$}\par
{\tt \#include $<$stdio.h$>$}\par
{\tt \#include $<$string.h$>$}\par
{\tt \#include $<$errno.h$>$}\par
{\tt \#include $<$sys/time.h$>$}\par
{\tt \#include $<$sys/socket.h$>$}\par
{\tt \#include $<$signal.h$>$}\par
{\tt \#include $<$fcntl.h$>$}\par
{\tt \#include \char`\"{}glib.h\char`\"{}}\par
\subsection*{Data Structures}
\begin{CompactItemize}
\item 
struct \hyperlink{structlisten__source__t}{listen\_\-source\_\-t}
\begin{CompactList}\small\item\em listen source state \item\end{CompactList}\item 
struct \hyperlink{structecho__source__t}{echo\_\-source\_\-t}
\begin{CompactList}\small\item\em echo source state \item\end{CompactList}\item 
struct \hyperlink{structchargen__source__t}{chargen\_\-source\_\-t}
\begin{CompactList}\small\item\em chargen source state \item\end{CompactList}\item 
struct \hyperlink{structsource__t}{source\_\-t}
\end{CompactItemize}
\subsection*{Defines}
\begin{CompactItemize}
\item 
\#define \hyperlink{mainloop__glib_8c_a0}{PORT\_\-ECHO}\ 5005
\begin{CompactList}\small\item\em tcp port to listen for echo clients \item\end{CompactList}\item 
\#define \hyperlink{mainloop__glib_8c_a1}{PORT\_\-CHARGEN}\ 5006
\begin{CompactList}\small\item\em tcp port to listen for chargen clients \item\end{CompactList}\item 
\#define \hyperlink{mainloop__glib_8c_a2}{HEARTBEAT\_\-INTERVAL}\ 2
\begin{CompactList}\small\item\em heartbeat interval (in seconds) \item\end{CompactList}\item 
\#define \hyperlink{mainloop__glib_8c_a3}{SLOWHEARTBEAT\_\-INTERVAL}\ 15
\begin{CompactList}\small\item\em slow heartbeat interval (in seconds) \item\end{CompactList}\item 
\#define \hyperlink{mainloop__glib_8c_a4}{BUFSIZE}\ 16
\begin{CompactList}\small\item\em buffer size for echo client \item\end{CompactList}\item 
\#define \hyperlink{mainloop__glib_8c_a5}{MSG\_\-HEARTBEAT}\ 0
\item 
\#define \hyperlink{mainloop__glib_8c_a6}{MSG\_\-SLOWHEARTBEAT}\ 1
\item 
\#define \hyperlink{mainloop__glib_8c_a7}{MSG\_\-MAINLOOP}\ 2
\item 
\#define \hyperlink{mainloop__glib_8c_a8}{MSG\_\-ACCEPT}\ 3
\item 
\#define \hyperlink{mainloop__glib_8c_a9}{MSG\_\-TOOMANY}\ 4
\item 
\#define \hyperlink{mainloop__glib_8c_a10}{MSG\_\-CLOSE}\ 5
\item 
\#define \hyperlink{mainloop__glib_8c_a11}{MSG\_\-READ}\ 6
\item 
\#define \hyperlink{mainloop__glib_8c_a12}{MSG\_\-WRITE}\ 7
\item 
\#define \hyperlink{mainloop__glib_8c_a13}{MSG\_\-FULL}\ 8
\item 
\#define \hyperlink{mainloop__glib_8c_a14}{MSG\_\-EMPTY}\ 9
\end{CompactItemize}
\subsection*{Typedefs}
\begin{CompactItemize}
\item 
typedef GSource $\ast$($\ast$ \hyperlink{mainloop__glib_8c_a16}{getclientsourcefunc} )()
\end{CompactItemize}
\subsection*{Functions}
\begin{CompactItemize}
\item 
void \hyperlink{mainloop__glib_8c_a17}{message} (int msg)
\begin{CompactList}\small\item\em print message describing current activity \item\end{CompactList}\item 
int \hyperlink{mainloop__glib_8c_a18}{listensocket} (int port)
\begin{CompactList}\small\item\em return tcp socket listening on port specified \item\end{CompactList}\item 
void \hyperlink{mainloop__glib_8c_a19}{setnonblock} (int fd)
\begin{CompactList}\small\item\em set a file descriptor to be nonblocking \item\end{CompactList}\item 
\hyperlink{structsource__t}{source\_\-t} $\ast$ \hyperlink{mainloop__glib_8c_a20}{listensource} (\hyperlink{mainloop__glib_8c_a16}{getclientsourcefunc} getclientsource, int port)
\begin{CompactList}\small\item\em construct listensource listening on port specified \item\end{CompactList}\item 
GSource $\ast$ \hyperlink{mainloop__glib_8c_a21}{getechoclientsource} ()
\begin{CompactList}\small\item\em construct GSource\-Funcs for echo client \item\end{CompactList}\item 
GSource $\ast$ \hyperlink{mainloop__glib_8c_a22}{getchargenclientsource} ()
\begin{CompactList}\small\item\em construct GSource\-Funcs for chargen client \item\end{CompactList}\item 
gboolean \hyperlink{mainloop__glib_8c_a23}{heartbeat} (gpointer data)
\begin{CompactList}\small\item\em print heartbeat message \item\end{CompactList}\item 
gboolean \hyperlink{mainloop__glib_8c_a24}{slowheartbeat} (gpointer data)
\begin{CompactList}\small\item\em print \char`\"{}slowheartbeat\char`\"{} message \item\end{CompactList}\item 
gboolean \hyperlink{mainloop__glib_8c_a25}{check} (GSource $\ast$source)
\begin{CompactList}\small\item\em check whether source is ready for processing \item\end{CompactList}\item 
void \hyperlink{mainloop__glib_8c_a26}{source\_\-close} (\hyperlink{structsource__t}{source\_\-t} $\ast$source)
\begin{CompactList}\small\item\em close a source and dispose of source \item\end{CompactList}\item 
gboolean \hyperlink{mainloop__glib_8c_a27}{accept\_\-prepare} (GSource $\ast$source, gint $\ast$timeout)
\begin{CompactList}\small\item\em Prepare GSource for polling a server socket for accept. \item\end{CompactList}\item 
gboolean \hyperlink{mainloop__glib_8c_a28}{accept\_\-dispatch} (GSource $\ast$source, GSource\-Func callback, gpointer user\_\-data)
\begin{CompactList}\small\item\em dispatch (process) listening socket \item\end{CompactList}\item 
gboolean \hyperlink{mainloop__glib_8c_a29}{echo\_\-prepare} (GSource $\ast$source, gint $\ast$timeout)
\begin{CompactList}\small\item\em Prepare GSource for polling an echo client. \item\end{CompactList}\item 
gboolean \hyperlink{mainloop__glib_8c_a30}{echo\_\-dispatch} (GSource $\ast$source, GSource\-Func callback, gpointer user\_\-data)
\begin{CompactList}\small\item\em dispatch (process) echo client \item\end{CompactList}\item 
void \hyperlink{mainloop__glib_8c_a31}{echo\_\-dispatch\_\-read} (\hyperlink{structsource__t}{source\_\-t} $\ast$echosource)
\begin{CompactList}\small\item\em read data from echo client \item\end{CompactList}\item 
void \hyperlink{mainloop__glib_8c_a32}{echo\_\-dispatch\_\-write} (\hyperlink{structsource__t}{source\_\-t} $\ast$echosource)
\begin{CompactList}\small\item\em write data to an echo client \item\end{CompactList}\item 
gboolean \hyperlink{mainloop__glib_8c_a33}{chargen\_\-prepare} (GSource $\ast$source, gint $\ast$timeout)
\begin{CompactList}\small\item\em Prepare GSource for polling an chargen client. \item\end{CompactList}\item 
gboolean \hyperlink{mainloop__glib_8c_a34}{chargen\_\-dispatch} (GSource $\ast$source, GSource\-Func callback, gpointer user\_\-data)
\begin{CompactList}\small\item\em dispatch (process) chargen client \item\end{CompactList}\item 
void \hyperlink{mainloop__glib_8c_a35}{blocksigpipe} ()
\begin{CompactList}\small\item\em block SIGPIPE \item\end{CompactList}\item 
int \hyperlink{mainloop__glib_8c_a36}{main} ()
\end{CompactItemize}
\subsection*{Variables}
\begin{CompactItemize}
\item 
char \hyperlink{mainloop__glib_8c_a15}{chargen\_\-buf} \mbox{[}$\,$\mbox{]} = \char`\"{}0123456789abcdefghijklmnopqrstuv\char`\"{}
\begin{CompactList}\small\item\em characters to return in chargen service \item\end{CompactList}\end{CompactItemize}


\subsection{Detailed Description}
Example for main loop using glib 2. 

\begin{Desc}
\item[Author:]Rico Pajarola\end{Desc}
This example uses glib 2 for handling events.

Definition in file \hyperlink{mainloop__glib_8c-source}{mainloop\_\-glib.c}.

\subsection{Define Documentation}
\hypertarget{mainloop__glib_8c_a4}{
\index{mainloop_glib.c@{mainloop\_\-glib.c}!BUFSIZE@{BUFSIZE}}
\index{BUFSIZE@{BUFSIZE}!mainloop_glib.c@{mainloop\_\-glib.c}}
\subsubsection[BUFSIZE]{\setlength{\rightskip}{0pt plus 5cm}\#define BUFSIZE\ 16}}
\label{mainloop__glib_8c_a4}


buffer size for echo client 



Definition at line 35 of file mainloop\_\-glib.c.\hypertarget{mainloop__glib_8c_a2}{
\index{mainloop_glib.c@{mainloop\_\-glib.c}!HEARTBEAT_INTERVAL@{HEARTBEAT\_\-INTERVAL}}
\index{HEARTBEAT_INTERVAL@{HEARTBEAT\_\-INTERVAL}!mainloop_glib.c@{mainloop\_\-glib.c}}
\subsubsection[HEARTBEAT\_\-INTERVAL]{\setlength{\rightskip}{0pt plus 5cm}\#define HEARTBEAT\_\-INTERVAL\ 2}}
\label{mainloop__glib_8c_a2}


heartbeat interval (in seconds) 



Definition at line 29 of file mainloop\_\-glib.c.\hypertarget{mainloop__glib_8c_a8}{
\index{mainloop_glib.c@{mainloop\_\-glib.c}!MSG_ACCEPT@{MSG\_\-ACCEPT}}
\index{MSG_ACCEPT@{MSG\_\-ACCEPT}!mainloop_glib.c@{mainloop\_\-glib.c}}
\subsubsection[MSG\_\-ACCEPT]{\setlength{\rightskip}{0pt plus 5cm}\#define MSG\_\-ACCEPT\ 3}}
\label{mainloop__glib_8c_a8}




Definition at line 40 of file mainloop\_\-glib.c.\hypertarget{mainloop__glib_8c_a10}{
\index{mainloop_glib.c@{mainloop\_\-glib.c}!MSG_CLOSE@{MSG\_\-CLOSE}}
\index{MSG_CLOSE@{MSG\_\-CLOSE}!mainloop_glib.c@{mainloop\_\-glib.c}}
\subsubsection[MSG\_\-CLOSE]{\setlength{\rightskip}{0pt plus 5cm}\#define MSG\_\-CLOSE\ 5}}
\label{mainloop__glib_8c_a10}




Definition at line 42 of file mainloop\_\-glib.c.\hypertarget{mainloop__glib_8c_a14}{
\index{mainloop_glib.c@{mainloop\_\-glib.c}!MSG_EMPTY@{MSG\_\-EMPTY}}
\index{MSG_EMPTY@{MSG\_\-EMPTY}!mainloop_glib.c@{mainloop\_\-glib.c}}
\subsubsection[MSG\_\-EMPTY]{\setlength{\rightskip}{0pt plus 5cm}\#define MSG\_\-EMPTY\ 9}}
\label{mainloop__glib_8c_a14}




Definition at line 46 of file mainloop\_\-glib.c.\hypertarget{mainloop__glib_8c_a13}{
\index{mainloop_glib.c@{mainloop\_\-glib.c}!MSG_FULL@{MSG\_\-FULL}}
\index{MSG_FULL@{MSG\_\-FULL}!mainloop_glib.c@{mainloop\_\-glib.c}}
\subsubsection[MSG\_\-FULL]{\setlength{\rightskip}{0pt plus 5cm}\#define MSG\_\-FULL\ 8}}
\label{mainloop__glib_8c_a13}




Definition at line 45 of file mainloop\_\-glib.c.\hypertarget{mainloop__glib_8c_a5}{
\index{mainloop_glib.c@{mainloop\_\-glib.c}!MSG_HEARTBEAT@{MSG\_\-HEARTBEAT}}
\index{MSG_HEARTBEAT@{MSG\_\-HEARTBEAT}!mainloop_glib.c@{mainloop\_\-glib.c}}
\subsubsection[MSG\_\-HEARTBEAT]{\setlength{\rightskip}{0pt plus 5cm}\#define MSG\_\-HEARTBEAT\ 0}}
\label{mainloop__glib_8c_a5}




Definition at line 37 of file mainloop\_\-glib.c.\hypertarget{mainloop__glib_8c_a7}{
\index{mainloop_glib.c@{mainloop\_\-glib.c}!MSG_MAINLOOP@{MSG\_\-MAINLOOP}}
\index{MSG_MAINLOOP@{MSG\_\-MAINLOOP}!mainloop_glib.c@{mainloop\_\-glib.c}}
\subsubsection[MSG\_\-MAINLOOP]{\setlength{\rightskip}{0pt plus 5cm}\#define MSG\_\-MAINLOOP\ 2}}
\label{mainloop__glib_8c_a7}




Definition at line 39 of file mainloop\_\-glib.c.\hypertarget{mainloop__glib_8c_a11}{
\index{mainloop_glib.c@{mainloop\_\-glib.c}!MSG_READ@{MSG\_\-READ}}
\index{MSG_READ@{MSG\_\-READ}!mainloop_glib.c@{mainloop\_\-glib.c}}
\subsubsection[MSG\_\-READ]{\setlength{\rightskip}{0pt plus 5cm}\#define MSG\_\-READ\ 6}}
\label{mainloop__glib_8c_a11}




Definition at line 43 of file mainloop\_\-glib.c.\hypertarget{mainloop__glib_8c_a6}{
\index{mainloop_glib.c@{mainloop\_\-glib.c}!MSG_SLOWHEARTBEAT@{MSG\_\-SLOWHEARTBEAT}}
\index{MSG_SLOWHEARTBEAT@{MSG\_\-SLOWHEARTBEAT}!mainloop_glib.c@{mainloop\_\-glib.c}}
\subsubsection[MSG\_\-SLOWHEARTBEAT]{\setlength{\rightskip}{0pt plus 5cm}\#define MSG\_\-SLOWHEARTBEAT\ 1}}
\label{mainloop__glib_8c_a6}




Definition at line 38 of file mainloop\_\-glib.c.\hypertarget{mainloop__glib_8c_a9}{
\index{mainloop_glib.c@{mainloop\_\-glib.c}!MSG_TOOMANY@{MSG\_\-TOOMANY}}
\index{MSG_TOOMANY@{MSG\_\-TOOMANY}!mainloop_glib.c@{mainloop\_\-glib.c}}
\subsubsection[MSG\_\-TOOMANY]{\setlength{\rightskip}{0pt plus 5cm}\#define MSG\_\-TOOMANY\ 4}}
\label{mainloop__glib_8c_a9}




Definition at line 41 of file mainloop\_\-glib.c.\hypertarget{mainloop__glib_8c_a12}{
\index{mainloop_glib.c@{mainloop\_\-glib.c}!MSG_WRITE@{MSG\_\-WRITE}}
\index{MSG_WRITE@{MSG\_\-WRITE}!mainloop_glib.c@{mainloop\_\-glib.c}}
\subsubsection[MSG\_\-WRITE]{\setlength{\rightskip}{0pt plus 5cm}\#define MSG\_\-WRITE\ 7}}
\label{mainloop__glib_8c_a12}




Definition at line 44 of file mainloop\_\-glib.c.\hypertarget{mainloop__glib_8c_a1}{
\index{mainloop_glib.c@{mainloop\_\-glib.c}!PORT_CHARGEN@{PORT\_\-CHARGEN}}
\index{PORT_CHARGEN@{PORT\_\-CHARGEN}!mainloop_glib.c@{mainloop\_\-glib.c}}
\subsubsection[PORT\_\-CHARGEN]{\setlength{\rightskip}{0pt plus 5cm}\#define PORT\_\-CHARGEN\ 5006}}
\label{mainloop__glib_8c_a1}


tcp port to listen for chargen clients 



Definition at line 26 of file mainloop\_\-glib.c.\hypertarget{mainloop__glib_8c_a0}{
\index{mainloop_glib.c@{mainloop\_\-glib.c}!PORT_ECHO@{PORT\_\-ECHO}}
\index{PORT_ECHO@{PORT\_\-ECHO}!mainloop_glib.c@{mainloop\_\-glib.c}}
\subsubsection[PORT\_\-ECHO]{\setlength{\rightskip}{0pt plus 5cm}\#define PORT\_\-ECHO\ 5005}}
\label{mainloop__glib_8c_a0}


tcp port to listen for echo clients 



Definition at line 23 of file mainloop\_\-glib.c.\hypertarget{mainloop__glib_8c_a3}{
\index{mainloop_glib.c@{mainloop\_\-glib.c}!SLOWHEARTBEAT_INTERVAL@{SLOWHEARTBEAT\_\-INTERVAL}}
\index{SLOWHEARTBEAT_INTERVAL@{SLOWHEARTBEAT\_\-INTERVAL}!mainloop_glib.c@{mainloop\_\-glib.c}}
\subsubsection[SLOWHEARTBEAT\_\-INTERVAL]{\setlength{\rightskip}{0pt plus 5cm}\#define SLOWHEARTBEAT\_\-INTERVAL\ 15}}
\label{mainloop__glib_8c_a3}


slow heartbeat interval (in seconds) 



Definition at line 32 of file mainloop\_\-glib.c.

\subsection{Typedef Documentation}
\hypertarget{mainloop__glib_8c_a16}{
\index{mainloop_glib.c@{mainloop\_\-glib.c}!getclientsourcefunc@{getclientsourcefunc}}
\index{getclientsourcefunc@{getclientsourcefunc}!mainloop_glib.c@{mainloop\_\-glib.c}}
\subsubsection[getclientsourcefunc]{\setlength{\rightskip}{0pt plus 5cm}typedef GSource$\ast$($\ast$ \hyperlink{mainloop__glib_8c_a16}{getclientsourcefunc})()}}
\label{mainloop__glib_8c_a16}




Definition at line 51 of file mainloop\_\-glib.c.

\subsection{Function Documentation}
\hypertarget{mainloop__glib_8c_a28}{
\index{mainloop_glib.c@{mainloop\_\-glib.c}!accept_dispatch@{accept\_\-dispatch}}
\index{accept_dispatch@{accept\_\-dispatch}!mainloop_glib.c@{mainloop\_\-glib.c}}
\subsubsection[accept\_\-dispatch]{\setlength{\rightskip}{0pt plus 5cm}gboolean accept\_\-dispatch (GSource $\ast$ {\em gsource}, GSource\-Func {\em callback}, gpointer {\em user\_\-data})\hspace{0.3cm}{\tt  \mbox{[}static\mbox{]}}}}
\label{mainloop__glib_8c_a28}


dispatch (process) listening socket 

\begin{Desc}
\item[Parameters:]
\begin{description}
\item[{\em gsource}]source to process \item[{\em callback}]callback function (unused) \item[{\em user\_\-data}]I have absolutely no idea how to use this... nice idea though\end{description}
\end{Desc}
\begin{Desc}
\item[Returns:]always TRUE \end{Desc}


Definition at line 347 of file mainloop\_\-glib.c.

References listen\_\-source\_\-t::getclientsource, source\_\-t::listen, message(), MSG\_\-ACCEPT, source\_\-t::pollfd, and setnonblock().

Referenced by listensource().



\footnotesize\begin{verbatim}349 {
350     struct sockaddr_in sain;
351     int             fd;
352     socklen_t       t;
353     source_t       *source;
354     source_t       *clientsource;
355 
356     source = (source_t *) gsource;
357 
358     if (source->pollfd.revents & (G_IO_HUP | G_IO_ERR)) {
359         perror("accept_dispatch()");
360         exit(EXIT_FAILURE);
361     }
362 
363     /* accept new connection */
364     t = sizeof(sain);
365     if ((fd = accept(source->pollfd.fd, (void *) &sain, &t)) == -1) {
366         if (errno == EWOULDBLOCK) {
367             return TRUE;
368         }
369         perror("accept(ECHO)");
370         exit(EXIT_FAILURE);
371     }
372     setnonblock(fd);
373     message(MSG_ACCEPT);
374 
375     clientsource = (source_t *) source->listen.getclientsource();
376     clientsource->pollfd.fd = fd;
377     g_source_add_poll((GSource *) clientsource, &(clientsource->pollfd));
378     clientsource->id =
379         g_source_attach((GSource *) clientsource,
380                         g_source_get_context((GSource *) clientsource));
381     return TRUE;
382 }
\end{verbatim}\normalsize 
\hypertarget{mainloop__glib_8c_a27}{
\index{mainloop_glib.c@{mainloop\_\-glib.c}!accept_prepare@{accept\_\-prepare}}
\index{accept_prepare@{accept\_\-prepare}!mainloop_glib.c@{mainloop\_\-glib.c}}
\subsubsection[accept\_\-prepare]{\setlength{\rightskip}{0pt plus 5cm}gboolean accept\_\-prepare (GSource $\ast$ {\em source}, gint $\ast$ {\em timeout})\hspace{0.3cm}{\tt  \mbox{[}static\mbox{]}}}}
\label{mainloop__glib_8c_a27}


Prepare GSource for polling a server socket for accept. 

\begin{Desc}
\item[Parameters:]
\begin{description}
\item[{\em source}]GSource to prepare \item[{\em timeout}]maximum timeout to set for poll() (out)\end{description}
\end{Desc}
\begin{Desc}
\item[Returns:]always FALSE (use poll) \end{Desc}


Definition at line 326 of file mainloop\_\-glib.c.

References source\_\-t::pollfd.

Referenced by listensource().



\footnotesize\begin{verbatim}327 {
328     source_t       *xsource;
329 
330     xsource = (source_t *) source;
331 
332     xsource->pollfd.events = G_IO_IN;
333     *timeout = -1;              /* no timeout */
334     return FALSE;
335 }
\end{verbatim}\normalsize 
\hypertarget{mainloop__glib_8c_a35}{
\index{mainloop_glib.c@{mainloop\_\-glib.c}!blocksigpipe@{blocksigpipe}}
\index{blocksigpipe@{blocksigpipe}!mainloop_glib.c@{mainloop\_\-glib.c}}
\subsubsection[blocksigpipe]{\setlength{\rightskip}{0pt plus 5cm}void blocksigpipe (void)\hspace{0.3cm}{\tt  \mbox{[}static\mbox{]}}}}
\label{mainloop__glib_8c_a35}


block SIGPIPE 

Trying to write to a socket when the other end has already closed the connection results in SIGPIPE. Not usefull in this context. 

Definition at line 170 of file mainloop\_\-glib.c.

Referenced by main().



\footnotesize\begin{verbatim}171 {
172     struct sigaction act;
173 
174     act.sa_handler = SIG_IGN;
175     sigemptyset(&act.sa_mask);
176     act.sa_flags = SA_RESTART;
177     if (sigaction(SIGPIPE, &act, NULL) == -1) {
178         perror("sigaction(SIGPIPE, <ignore>)");
179         exit(EXIT_FAILURE);
180     }
181 }
\end{verbatim}\normalsize 
\hypertarget{mainloop__glib_8c_a34}{
\index{mainloop_glib.c@{mainloop\_\-glib.c}!chargen_dispatch@{chargen\_\-dispatch}}
\index{chargen_dispatch@{chargen\_\-dispatch}!mainloop_glib.c@{mainloop\_\-glib.c}}
\subsubsection[chargen\_\-dispatch]{\setlength{\rightskip}{0pt plus 5cm}gboolean chargen\_\-dispatch (GSource $\ast$ {\em source}, GSource\-Func {\em callback}, gpointer {\em user\_\-data})\hspace{0.3cm}{\tt  \mbox{[}static\mbox{]}}}}
\label{mainloop__glib_8c_a34}


dispatch (process) chargen client 

\begin{Desc}
\item[Parameters:]
\begin{description}
\item[{\em source}]source to process \item[{\em callback}]callback function (unused) \item[{\em user\_\-data}]?\end{description}
\end{Desc}
\begin{Desc}
\item[Returns:]always TRUE\end{Desc}
there is no chargen\_\-dispatch\_\-write, writing is done directly in chargen\_\-dispatch 

Definition at line 633 of file mainloop\_\-glib.c.

References source\_\-t::chargen, chargen\_\-buf, chargen\_\-source\_\-t::i, message(), MSG\_\-WRITE, source\_\-t::pollfd, and source\_\-close().

Referenced by getchargenclientsource().



\footnotesize\begin{verbatim}635 {
636     ssize_t         nwrite;
637     source_t       *chargensource;
638 
639     chargensource = (source_t *) source;
640 
641     nwrite =
642         write(chargensource->pollfd.fd,
643               chargen_buf + chargensource->chargen.i,
644               sizeof(chargen_buf) - chargensource->chargen.i);
645     switch (nwrite) {
646     case -1:
647         if (errno == EPIPE) {
648             source_close((source_t *) chargensource);
649             return TRUE;
650         } else if ((errno != EINTR) && (errno != EWOULDBLOCK)) {
651             perror("write()");
652             exit(EXIT_FAILURE);
653         }
654         break;
655     case 0:
656         break;
657     default:
658         message(MSG_WRITE);
659         chargensource->chargen.i += nwrite;
660         chargensource->chargen.i %= sizeof(chargen_buf);
661     }
662     return TRUE;
663 }
\end{verbatim}\normalsize 
\hypertarget{mainloop__glib_8c_a33}{
\index{mainloop_glib.c@{mainloop\_\-glib.c}!chargen_prepare@{chargen\_\-prepare}}
\index{chargen_prepare@{chargen\_\-prepare}!mainloop_glib.c@{mainloop\_\-glib.c}}
\subsubsection[chargen\_\-prepare]{\setlength{\rightskip}{0pt plus 5cm}gboolean chargen\_\-prepare (GSource $\ast$ {\em source}, gint $\ast$ {\em timeout})\hspace{0.3cm}{\tt  \mbox{[}static\mbox{]}}}}
\label{mainloop__glib_8c_a33}


Prepare GSource for polling an chargen client. 

\begin{Desc}
\item[Parameters:]
\begin{description}
\item[{\em source}]GSource to prepare \item[{\em timeout}]maximum timeout to set for poll() (out)\end{description}
\end{Desc}
\begin{Desc}
\item[Returns:]always FALSE (use poll) \end{Desc}


Definition at line 609 of file mainloop\_\-glib.c.

References source\_\-t::pollfd.

Referenced by getchargenclientsource().



\footnotesize\begin{verbatim}610 {
611     source_t       *chargensource;
612 
613     chargensource = (source_t *) source;
614 
615     chargensource->pollfd.events = G_IO_OUT;
616 
617     return FALSE;
618 }
\end{verbatim}\normalsize 
\hypertarget{mainloop__glib_8c_a25}{
\index{mainloop_glib.c@{mainloop\_\-glib.c}!check@{check}}
\index{check@{check}!mainloop_glib.c@{mainloop\_\-glib.c}}
\subsubsection[check]{\setlength{\rightskip}{0pt plus 5cm}gboolean check (GSource $\ast$ {\em source})\hspace{0.3cm}{\tt  \mbox{[}static\mbox{]}}}}
\label{mainloop__glib_8c_a25}


check whether source is ready for processing 

\begin{Desc}
\item[Parameters:]
\begin{description}
\item[{\em source}]GSource to check\end{description}
\end{Desc}
\begin{Desc}
\item[Returns:]TRUE if resource is ready \end{Desc}


Definition at line 307 of file mainloop\_\-glib.c.

References source\_\-t::pollfd.

Referenced by getchargenclientsource(), getechoclientsource(), and listensource().



\footnotesize\begin{verbatim}308 {
309 
310     source_t       *xsource;
311 
312     xsource = (source_t *) source;
313 
314     return xsource->pollfd.revents ? TRUE : FALSE;
315 }
\end{verbatim}\normalsize 
\hypertarget{mainloop__glib_8c_a30}{
\index{mainloop_glib.c@{mainloop\_\-glib.c}!echo_dispatch@{echo\_\-dispatch}}
\index{echo_dispatch@{echo\_\-dispatch}!mainloop_glib.c@{mainloop\_\-glib.c}}
\subsubsection[echo\_\-dispatch]{\setlength{\rightskip}{0pt plus 5cm}gboolean echo\_\-dispatch (GSource $\ast$ {\em source}, GSource\-Func {\em callback}, gpointer {\em user\_\-data})\hspace{0.3cm}{\tt  \mbox{[}static\mbox{]}}}}
\label{mainloop__glib_8c_a30}


dispatch (process) echo client 

\begin{Desc}
\item[Parameters:]
\begin{description}
\item[{\em source}]source to process \item[{\em callback}]callback function (unused) \item[{\em user\_\-data}]?\end{description}
\end{Desc}
\begin{Desc}
\item[Returns:]always TRUE \end{Desc}


Definition at line 471 of file mainloop\_\-glib.c.

References echo\_\-dispatch\_\-read(), echo\_\-dispatch\_\-write(), source\_\-t::pollfd, and source\_\-close().

Referenced by getechoclientsource().



\footnotesize\begin{verbatim}472 {
473     source_t       *echosource;
474 
475     echosource = (source_t *) source;
476 
477     if (echosource->pollfd.revents & (G_IO_HUP | G_IO_ERR)) {
478         source_close((source_t *) echosource);
479     }
480     if (echosource->pollfd.revents & G_IO_IN) {
481         echo_dispatch_read(echosource);
482     }
483     if (echosource->pollfd.revents & G_IO_OUT) {
484         echo_dispatch_write(echosource);
485     }
486     return TRUE;
487 }
\end{verbatim}\normalsize 
\hypertarget{mainloop__glib_8c_a31}{
\index{mainloop_glib.c@{mainloop\_\-glib.c}!echo_dispatch_read@{echo\_\-dispatch\_\-read}}
\index{echo_dispatch_read@{echo\_\-dispatch\_\-read}!mainloop_glib.c@{mainloop\_\-glib.c}}
\subsubsection[echo\_\-dispatch\_\-read]{\setlength{\rightskip}{0pt plus 5cm}void echo\_\-dispatch\_\-read (\hyperlink{structsource__t}{source\_\-t} $\ast$ {\em echosource})\hspace{0.3cm}{\tt  \mbox{[}static\mbox{]}}}}
\label{mainloop__glib_8c_a31}


read data from echo client 

\begin{Desc}
\item[Parameters:]
\begin{description}
\item[{\em echosource}]echo client source\end{description}
\end{Desc}
If the buffer is not full, tries to do one read from the filedescriptor associated with this echo client. 

Definition at line 498 of file mainloop\_\-glib.c.

References echo\_\-source\_\-t::buf, BUFSIZE, source\_\-t::echo, message(), MSG\_\-FULL, MSG\_\-READ, echo\_\-source\_\-t::n, source\_\-t::pollfd, echo\_\-source\_\-t::r, source\_\-close(), and echo\_\-source\_\-t::w.

Referenced by echo\_\-dispatch().



\footnotesize\begin{verbatim}499 {
500     ssize_t         nread;
501 
502     if (echosource->echo.n == BUFSIZE) {
503         message(MSG_FULL);
504         return;
505     }
506 
507     if (echosource->echo.r >= echosource->echo.w) {
508         nread =
509             read(echosource->pollfd.fd,
510                  echosource->echo.buf + echosource->echo.r,
511                  BUFSIZE - echosource->echo.r);
512     } else {
513         nread =
514             read(echosource->pollfd.fd,
515                  echosource->echo.buf + echosource->echo.r,
516                  echosource->echo.w - echosource->echo.r);
517     }
518 
519     switch (nread) {
520     case -1:
521         if ((errno != EINTR) && (errno != EWOULDBLOCK)) {
522             perror("read()");
523             exit(EXIT_FAILURE);
524         }
525         break;
526     case 0:
527         source_close((source_t *) echosource);
528         return;
529     default:
530         message(MSG_READ);
531         echosource->echo.n += nread;
532         echosource->echo.r += nread;
533         echosource->echo.r %= BUFSIZE;
534     }
535 }
\end{verbatim}\normalsize 
\hypertarget{mainloop__glib_8c_a32}{
\index{mainloop_glib.c@{mainloop\_\-glib.c}!echo_dispatch_write@{echo\_\-dispatch\_\-write}}
\index{echo_dispatch_write@{echo\_\-dispatch\_\-write}!mainloop_glib.c@{mainloop\_\-glib.c}}
\subsubsection[echo\_\-dispatch\_\-write]{\setlength{\rightskip}{0pt plus 5cm}void echo\_\-dispatch\_\-write (\hyperlink{structsource__t}{source\_\-t} $\ast$ {\em echosource})\hspace{0.3cm}{\tt  \mbox{[}static\mbox{]}}}}
\label{mainloop__glib_8c_a32}


write data to an echo client 

\begin{Desc}
\item[Parameters:]
\begin{description}
\item[{\em echosource}]echo client source\end{description}
\end{Desc}
If the buffer is not empty, tries to do one write to the filedescriptor associated with the echo client. 

Definition at line 546 of file mainloop\_\-glib.c.

References echo\_\-source\_\-t::buf, BUFSIZE, source\_\-t::echo, message(), MSG\_\-EMPTY, MSG\_\-WRITE, echo\_\-source\_\-t::n, source\_\-t::pollfd, echo\_\-source\_\-t::r, source\_\-close(), and echo\_\-source\_\-t::w.

Referenced by echo\_\-dispatch().



\footnotesize\begin{verbatim}547 {
548     ssize_t         nwrite;
549 
550     if (echosource->echo.n == 0) {
551         message(MSG_EMPTY);
552         return;
553     }
554 
555     if (echosource->echo.r > echosource->echo.w) {
556         nwrite =
557             write(echosource->pollfd.fd,
558                   echosource->echo.buf + echosource->echo.w,
559                   echosource->echo.r - echosource->echo.w);
560     } else {
561         nwrite =
562             write(echosource->pollfd.fd,
563                   echosource->echo.buf + echosource->echo.w,
564                   BUFSIZE - echosource->echo.w);
565     }
566 
567     switch (nwrite) {
568     case -1:
569         if (errno == EPIPE) {
570             source_close((source_t *) echosource);
571         } else if ((errno != EINTR) && (errno != EWOULDBLOCK)) {
572             perror("write()");
573             exit(EXIT_FAILURE);
574         }
575         break;
576     case 0:
577         break;
578     default:
579         message(MSG_WRITE);
580         echosource->echo.n -= nwrite;
581         echosource->echo.w += nwrite;
582         echosource->echo.w %= BUFSIZE;
583     }
584 }
\end{verbatim}\normalsize 
\hypertarget{mainloop__glib_8c_a29}{
\index{mainloop_glib.c@{mainloop\_\-glib.c}!echo_prepare@{echo\_\-prepare}}
\index{echo_prepare@{echo\_\-prepare}!mainloop_glib.c@{mainloop\_\-glib.c}}
\subsubsection[echo\_\-prepare]{\setlength{\rightskip}{0pt plus 5cm}gboolean echo\_\-prepare (GSource $\ast$ {\em source}, gint $\ast$ {\em timeout})\hspace{0.3cm}{\tt  \mbox{[}static\mbox{]}}}}
\label{mainloop__glib_8c_a29}


Prepare GSource for polling an echo client. 

\begin{Desc}
\item[Parameters:]
\begin{description}
\item[{\em source}]GSource to prepare \item[{\em timeout}]maximum timeout to set for poll() (out)\end{description}
\end{Desc}
\begin{Desc}
\item[Returns:]always FALSE (use poll)\end{Desc}
If the buffer is full, checking for read is turned off resp. if the buffer is empty, checking for write is turned off. 

Definition at line 442 of file mainloop\_\-glib.c.

References BUFSIZE, source\_\-t::echo, echo\_\-source\_\-t::n, and source\_\-t::pollfd.

Referenced by getechoclientsource().



\footnotesize\begin{verbatim}443 {
444     source_t       *echosource;
445 
446     echosource = (source_t *) source;
447 
448     switch (echosource->echo.n) {
449     case 0:
450         echosource->pollfd.events = G_IO_IN;
451         break;
452     case BUFSIZE:
453         echosource->pollfd.events = G_IO_OUT;
454         break;
455     default:
456         echosource->pollfd.events = G_IO_IN | G_IO_OUT;
457     }
458     return FALSE;
459 }
\end{verbatim}\normalsize 
\hypertarget{mainloop__glib_8c_a22}{
\index{mainloop_glib.c@{mainloop\_\-glib.c}!getchargenclientsource@{getchargenclientsource}}
\index{getchargenclientsource@{getchargenclientsource}!mainloop_glib.c@{mainloop\_\-glib.c}}
\subsubsection[getchargenclientsource]{\setlength{\rightskip}{0pt plus 5cm}GSource $\ast$ getchargenclientsource ()\hspace{0.3cm}{\tt  \mbox{[}static\mbox{]}}}}
\label{mainloop__glib_8c_a22}


construct GSource\-Funcs for chargen client 

\begin{Desc}
\item[Returns:]GSource for chargen client \end{Desc}


Definition at line 414 of file mainloop\_\-glib.c.

References source\_\-t::chargen, chargen\_\-dispatch(), chargen\_\-prepare(), check(), and chargen\_\-source\_\-t::i.

Referenced by main().



\footnotesize\begin{verbatim}415 {
416     static GSourceFuncs funcs = { NULL, NULL, NULL, NULL, NULL, NULL };
417     source_t       *chargensource;
418 
419     /* cannot reference functions in static initializer */
420     funcs.prepare = chargen_prepare;
421     funcs.check = check;
422     funcs.dispatch = chargen_dispatch;
423     funcs.finalize = NULL;
424 
425     chargensource = (source_t *) g_source_new(&funcs, sizeof(source_t));
426     chargensource->chargen.i = 0;
427     return (GSource *) chargensource;
428 }
\end{verbatim}\normalsize 
\hypertarget{mainloop__glib_8c_a21}{
\index{mainloop_glib.c@{mainloop\_\-glib.c}!getechoclientsource@{getechoclientsource}}
\index{getechoclientsource@{getechoclientsource}!mainloop_glib.c@{mainloop\_\-glib.c}}
\subsubsection[getechoclientsource]{\setlength{\rightskip}{0pt plus 5cm}GSource $\ast$ getechoclientsource ()\hspace{0.3cm}{\tt  \mbox{[}static\mbox{]}}}}
\label{mainloop__glib_8c_a21}


construct GSource\-Funcs for echo client 

\begin{Desc}
\item[Returns:]GSource for echo client \end{Desc}


Definition at line 390 of file mainloop\_\-glib.c.

References check(), source\_\-t::echo, echo\_\-dispatch(), echo\_\-prepare(), echo\_\-source\_\-t::n, echo\_\-source\_\-t::r, and echo\_\-source\_\-t::w.

Referenced by main().



\footnotesize\begin{verbatim}391 {
392     static GSourceFuncs funcs = { NULL, NULL, NULL, NULL, NULL, NULL };
393     source_t       *echosource;
394 
395     /* cannot reference functions in static initializer */
396     funcs.prepare = echo_prepare;
397     funcs.check = check;
398     funcs.dispatch = echo_dispatch;
399     funcs.finalize = NULL;
400 
401     echosource = (source_t *) g_source_new(&funcs, sizeof(source_t));
402     echosource->echo.r = 0;
403     echosource->echo.w = 0;
404     echosource->echo.n = 0;
405     return (GSource *) echosource;
406 }
\end{verbatim}\normalsize 
\hypertarget{mainloop__glib_8c_a23}{
\index{mainloop_glib.c@{mainloop\_\-glib.c}!heartbeat@{heartbeat}}
\index{heartbeat@{heartbeat}!mainloop_glib.c@{mainloop\_\-glib.c}}
\subsubsection[heartbeat]{\setlength{\rightskip}{0pt plus 5cm}gboolean heartbeat (gpointer {\em data})\hspace{0.3cm}{\tt  \mbox{[}static\mbox{]}}}}
\label{mainloop__glib_8c_a23}


print heartbeat message 

\begin{Desc}
\item[Returns:]always TRUE \end{Desc}


Definition at line 281 of file mainloop\_\-glib.c.

References message(), and MSG\_\-HEARTBEAT.

Referenced by main().



\footnotesize\begin{verbatim}282 {
283     message(MSG_HEARTBEAT);
284     return TRUE;
285 }
\end{verbatim}\normalsize 
\hypertarget{mainloop__glib_8c_a18}{
\index{mainloop_glib.c@{mainloop\_\-glib.c}!listensocket@{listensocket}}
\index{listensocket@{listensocket}!mainloop_glib.c@{mainloop\_\-glib.c}}
\subsubsection[listensocket]{\setlength{\rightskip}{0pt plus 5cm}int listensocket (int {\em port})\hspace{0.3cm}{\tt  \mbox{[}static\mbox{]}}}}
\label{mainloop__glib_8c_a18}


return tcp socket listening on port specified 

\begin{Desc}
\item[Parameters:]
\begin{description}
\item[{\em port}]port number in host byte order\end{description}
\end{Desc}
\begin{Desc}
\item[Returns:]file descriptor for listening socket\end{Desc}
The socket is set to be nonblocking 

Definition at line 193 of file mainloop\_\-glib.c.

References setnonblock().



\footnotesize\begin{verbatim}194 {
195     int             serverfd;
196     struct sockaddr_in sain;
197     int             one;
198 
199     /* set up tcp socket for listening */
200     sain.sin_family = AF_INET;
201     sain.sin_port = htons(port);
202     sain.sin_addr.s_addr = INADDR_ANY;
203     if ((serverfd = socket(AF_INET, SOCK_STREAM, 0)) == -1) {
204         perror("socket(AF_INET, SOCK_STREAM, 0)");
205         exit(EXIT_FAILURE);
206     }
207     one = 1;
208     if (setsockopt
209         (serverfd, SOL_SOCKET, SO_REUSEADDR, &one,
210          (int) sizeof(one)) == -1) {
211         perror("setsockopt(SO_REUSEADDR)");
212         exit(EXIT_FAILURE);
213     }
214     if (bind
215         (serverfd, (struct sockaddr *) &sain,
216          sizeof(struct sockaddr_in)) == -1) {
217         perror("bind()");
218         exit(EXIT_FAILURE);
219     }
220     if (listen(serverfd, 5) == -1) {
221         perror("listen()");
222         exit(EXIT_FAILURE);
223     }
224     setnonblock(serverfd);
225 
226     printf("listening on port %d\n", port);
227 
228     return serverfd;
229 }
\end{verbatim}\normalsize 
\hypertarget{mainloop__glib_8c_a20}{
\index{mainloop_glib.c@{mainloop\_\-glib.c}!listensource@{listensource}}
\index{listensource@{listensource}!mainloop_glib.c@{mainloop\_\-glib.c}}
\subsubsection[listensource]{\setlength{\rightskip}{0pt plus 5cm}\hyperlink{structsource__t}{source\_\-t} $\ast$ listensource (\hyperlink{mainloop__glib_8c_a16}{getclientsourcefunc} {\em getclientsource}, int {\em port})\hspace{0.3cm}{\tt  \mbox{[}static\mbox{]}}}}
\label{mainloop__glib_8c_a20}


construct listensource listening on port specified 

\begin{Desc}
\item[Parameters:]
\begin{description}
\item[{\em getclientsource}]function returning clientsource for accepted connection \item[{\em port}]tcp port to listen on in host byte order\end{description}
\end{Desc}
\begin{Desc}
\item[Returns:]listensource\_\-t \end{Desc}


Definition at line 257 of file mainloop\_\-glib.c.

References accept\_\-dispatch(), accept\_\-prepare(), check(), listen\_\-source\_\-t::getclientsource, source\_\-t::listen, listensocket(), listensource(), and source\_\-t::pollfd.

Referenced by listensource(), and main().



\footnotesize\begin{verbatim}258 {
259     static GSourceFuncs listenfuncs;
260     source_t       *listensource;
261 
262     listenfuncs.prepare = &accept_prepare;
263     listenfuncs.check = &check;
264     listenfuncs.dispatch = &accept_dispatch;
265     listenfuncs.finalize = NULL;
266     listensource =
267         (source_t *) g_source_new(&listenfuncs, sizeof(source_t));
268     listensource->listen.getclientsource = getclientsource;
269 
270     listensource->pollfd.fd = listensocket(port);
271     g_source_add_poll((GSource *) listensource, &(listensource->pollfd));
272     return listensource;
273 }
\end{verbatim}\normalsize 
\hypertarget{mainloop__glib_8c_a36}{
\index{mainloop_glib.c@{mainloop\_\-glib.c}!main@{main}}
\index{main@{main}!mainloop_glib.c@{mainloop\_\-glib.c}}
\subsubsection[main]{\setlength{\rightskip}{0pt plus 5cm}int main ()}}
\label{mainloop__glib_8c_a36}




Definition at line 666 of file mainloop\_\-glib.c.

References blocksigpipe(), getchargenclientsource(), getechoclientsource(), heartbeat(), HEARTBEAT\_\-INTERVAL, source\_\-t::id, listensource(), PORT\_\-CHARGEN, PORT\_\-ECHO, slowheartbeat(), and SLOWHEARTBEAT\_\-INTERVAL.



\footnotesize\begin{verbatim}667 {
668     GMainLoop      *mainloop;
669     GMainContext   *context;
670     source_t       *listenecho, *listenchargen;
671 
672     /* check glib version */
673     if (!GLIB_CHECK_VERSION(2, 0, 0)) {
674         fprintf(stderr, "glib %d.%d.%d is too old\n", GLIB_MAJOR_VERSION,
675                 GLIB_MINOR_VERSION, GLIB_MICRO_VERSION);
676         exit(EXIT_FAILURE);
677     }
678 
679     printf("example: glib main loop\n");
680 
681     blocksigpipe();
682 
683     /* create main loop */
684     context = g_main_context_default();
685     mainloop = g_main_loop_new(context, FALSE);
686 
687     /* create echo service */
688     listenecho = listensource(getechoclientsource, PORT_ECHO);
689     listenecho->id = g_source_attach((GSource *) listenecho, context);
690 
691     /* create chargen service */
692     listenchargen = listensource(getchargenclientsource, PORT_CHARGEN);
693     listenchargen->id =
694         g_source_attach((GSource *) listenchargen, context);
695 
696     /* install heartbeat */
697     (void) g_timeout_add(1000 * HEARTBEAT_INTERVAL, &heartbeat, NULL);
698     (void) g_timeout_add(1000 * SLOWHEARTBEAT_INTERVAL, &slowheartbeat, NULL);
699     printf("heartbeat every %d seconds\n", HEARTBEAT_INTERVAL);
700     printf("slow heartbeat every %d seconds\n", SLOWHEARTBEAT_INTERVAL);
701 
702     /* run the main loop */
703     g_main_loop_run(mainloop);
704     /* notreached */
705 
706     g_main_loop_unref(mainloop);
707     exit(EXIT_SUCCESS);
708 }
\end{verbatim}\normalsize 
\hypertarget{mainloop__glib_8c_a17}{
\index{mainloop_glib.c@{mainloop\_\-glib.c}!message@{message}}
\index{message@{message}!mainloop_glib.c@{mainloop\_\-glib.c}}
\subsubsection[message]{\setlength{\rightskip}{0pt plus 5cm}void message (int {\em msg})\hspace{0.3cm}{\tt  \mbox{[}static\mbox{]}}}}
\label{mainloop__glib_8c_a17}


print message describing current activity 

\begin{Desc}
\item[Parameters:]
\begin{description}
\item[{\em msg}]id of message to print (MSG\_\-XYZ) \end{description}
\end{Desc}


Definition at line 126 of file mainloop\_\-glib.c.

References MSG\_\-ACCEPT, MSG\_\-CLOSE, MSG\_\-EMPTY, MSG\_\-FULL, MSG\_\-HEARTBEAT, MSG\_\-MAINLOOP, MSG\_\-READ, MSG\_\-SLOWHEARTBEAT, MSG\_\-TOOMANY, and MSG\_\-WRITE.



\footnotesize\begin{verbatim}127 {
128     switch (msg) {
129     case MSG_HEARTBEAT:
130         printf("H");
131         break;
132     case MSG_SLOWHEARTBEAT:
133         printf("S");
134         break;
135     case MSG_MAINLOOP:
136         printf("M");
137         break;
138     case MSG_ACCEPT:
139         printf("A");
140         break;
141     case MSG_TOOMANY:
142         printf("T");
143         break;
144     case MSG_CLOSE:
145         printf("C");
146         break;
147     case MSG_READ:
148         printf("R");
149         break;
150     case MSG_WRITE:
151         printf("W");
152         break;
153     case MSG_FULL:
154         printf("F");
155         break;
156     case MSG_EMPTY:
157         printf("E");
158         break;
159     }
160     fflush(stdout);
161 }
\end{verbatim}\normalsize 
\hypertarget{mainloop__glib_8c_a19}{
\index{mainloop_glib.c@{mainloop\_\-glib.c}!setnonblock@{setnonblock}}
\index{setnonblock@{setnonblock}!mainloop_glib.c@{mainloop\_\-glib.c}}
\subsubsection[setnonblock]{\setlength{\rightskip}{0pt plus 5cm}void setnonblock (int {\em fd})\hspace{0.3cm}{\tt  \mbox{[}static\mbox{]}}}}
\label{mainloop__glib_8c_a19}


set a file descriptor to be nonblocking 

\begin{Desc}
\item[Parameters:]
\begin{description}
\item[{\em fd}]file descriptor\end{description}
\end{Desc}
Non-Blocking works only for Sockets, Pipes and slow devices, it has no effect when used with regular files 

Definition at line 240 of file mainloop\_\-glib.c.



\footnotesize\begin{verbatim}241 {
242     int             flag;
243 
244     flag = fcntl(fd, F_GETFL);
245     fcntl(fd, F_GETFL, flag | O_NONBLOCK);
246 }
\end{verbatim}\normalsize 
\hypertarget{mainloop__glib_8c_a24}{
\index{mainloop_glib.c@{mainloop\_\-glib.c}!slowheartbeat@{slowheartbeat}}
\index{slowheartbeat@{slowheartbeat}!mainloop_glib.c@{mainloop\_\-glib.c}}
\subsubsection[slowheartbeat]{\setlength{\rightskip}{0pt plus 5cm}gboolean slowheartbeat (gpointer {\em data})\hspace{0.3cm}{\tt  \mbox{[}static\mbox{]}}}}
\label{mainloop__glib_8c_a24}


print \char`\"{}slowheartbeat\char`\"{} message 

\begin{Desc}
\item[Returns:]always TRUE \end{Desc}


Definition at line 293 of file mainloop\_\-glib.c.

References message(), and MSG\_\-SLOWHEARTBEAT.

Referenced by main().



\footnotesize\begin{verbatim}294 {
295     message(MSG_SLOWHEARTBEAT);
296     return TRUE;
297 }
\end{verbatim}\normalsize 
\hypertarget{mainloop__glib_8c_a26}{
\index{mainloop_glib.c@{mainloop\_\-glib.c}!source_close@{source\_\-close}}
\index{source_close@{source\_\-close}!mainloop_glib.c@{mainloop\_\-glib.c}}
\subsubsection[source\_\-close]{\setlength{\rightskip}{0pt plus 5cm}void source\_\-close (\hyperlink{structsource__t}{source\_\-t} $\ast$ {\em source})\hspace{0.3cm}{\tt  \mbox{[}static\mbox{]}}}}
\label{mainloop__glib_8c_a26}


close a source and dispose of source 

\begin{Desc}
\item[Parameters:]
\begin{description}
\item[{\em source}]source to close/dispose \end{description}
\end{Desc}


Definition at line 592 of file mainloop\_\-glib.c.

References source\_\-t::id, message(), MSG\_\-CLOSE, and source\_\-t::pollfd.

Referenced by chargen\_\-dispatch(), echo\_\-dispatch(), echo\_\-dispatch\_\-read(), and echo\_\-dispatch\_\-write().



\footnotesize\begin{verbatim}593 {
594     message(MSG_CLOSE);
595     (void) g_source_remove(source->id);
596     close(source->pollfd.fd);
597     g_source_unref((GSource *) source);
598 }
\end{verbatim}\normalsize 


\subsection{Variable Documentation}
\hypertarget{mainloop__glib_8c_a15}{
\index{mainloop_glib.c@{mainloop\_\-glib.c}!chargen_buf@{chargen\_\-buf}}
\index{chargen_buf@{chargen\_\-buf}!mainloop_glib.c@{mainloop\_\-glib.c}}
\subsubsection[chargen\_\-buf]{\setlength{\rightskip}{0pt plus 5cm}char \hyperlink{mainloop__good_8c_a19}{chargen\_\-buf}\mbox{[}$\,$\mbox{]} = \char`\"{}0123456789abcdefghijklmnopqrstuv\char`\"{}\hspace{0.3cm}{\tt  \mbox{[}static\mbox{]}}}}
\label{mainloop__glib_8c_a15}


characters to return in chargen service 



Definition at line 49 of file mainloop\_\-glib.c.

Referenced by chargen\_\-dispatch().