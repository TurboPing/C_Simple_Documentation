\hypertarget{mainloop__good_8c}{
\section{mainloop\_\-good.c File Reference}
\label{mainloop__good_8c}\index{mainloop_good.c@{mainloop\_\-good.c}}
}
Example for \char`\"{}good\char`\"{} main loop.  


{\tt \#include $<$unistd.h$>$}\par
{\tt \#include $<$stdlib.h$>$}\par
{\tt \#include $<$stdio.h$>$}\par
{\tt \#include $<$string.h$>$}\par
{\tt \#include $<$errno.h$>$}\par
{\tt \#include $<$sys/time.h$>$}\par
{\tt \#include $<$sys/socket.h$>$}\par
{\tt \#include $<$signal.h$>$}\par
{\tt \#include $<$fcntl.h$>$}\par
\subsection*{Data Structures}
\begin{CompactItemize}
\item 
struct \hyperlink{structecho__client__t}{echo\_\-client\_\-t}
\begin{CompactList}\small\item\em echo client specific state \item\end{CompactList}\item 
struct \hyperlink{structchargen__client__t}{chargen\_\-client\_\-t}
\begin{CompactList}\small\item\em chargen client specific state \item\end{CompactList}\item 
struct \hyperlink{structclient__t}{client\_\-t}
\begin{CompactList}\small\item\em client state \item\end{CompactList}\item 
struct \hyperlink{structalarm__t}{alarm\_\-t}
\begin{CompactList}\small\item\em alarm handler state \item\end{CompactList}\end{CompactItemize}
\subsection*{Defines}
\begin{CompactItemize}
\item 
\#define \hyperlink{mainloop__good_8c_a0}{PORT\_\-ECHO}\ 5005
\begin{CompactList}\small\item\em tcp port to listen for echo clients \item\end{CompactList}\item 
\#define \hyperlink{mainloop__good_8c_a1}{PORT\_\-CHARGEN}\ 5006
\begin{CompactList}\small\item\em tcp port to listen for chargen clients \item\end{CompactList}\item 
\#define \hyperlink{mainloop__good_8c_a2}{HEARTBEAT\_\-INTERVAL}\ 2
\begin{CompactList}\small\item\em heartbeat interval (in seconds) \item\end{CompactList}\item 
\#define \hyperlink{mainloop__good_8c_a3}{SLOWHEARTBEAT\_\-INTERVAL}\ 15
\begin{CompactList}\small\item\em slow heartbeat interval (in seconds) \item\end{CompactList}\item 
\#define \hyperlink{mainloop__good_8c_a4}{BUFSIZE}\ 16
\begin{CompactList}\small\item\em buffer size for echo client \item\end{CompactList}\item 
\#define \hyperlink{mainloop__good_8c_a5}{MAXCLIENTS}\ 8
\begin{CompactList}\small\item\em maximum number of clients \item\end{CompactList}\item 
\#define \hyperlink{mainloop__good_8c_a6}{MAXALARMS}\ 8
\begin{CompactList}\small\item\em maximum number of different alarms \item\end{CompactList}\item 
\#define \hyperlink{mainloop__good_8c_a7}{MSG\_\-HEARTBEAT}\ 0
\item 
\#define \hyperlink{mainloop__good_8c_a8}{MSG\_\-SLOWHEARTBEAT}\ 1
\item 
\#define \hyperlink{mainloop__good_8c_a9}{MSG\_\-MAINLOOP}\ 2
\item 
\#define \hyperlink{mainloop__good_8c_a10}{MSG\_\-ACCEPT}\ 3
\item 
\#define \hyperlink{mainloop__good_8c_a11}{MSG\_\-TOOMANY}\ 4
\item 
\#define \hyperlink{mainloop__good_8c_a12}{MSG\_\-CLOSE}\ 5
\item 
\#define \hyperlink{mainloop__good_8c_a13}{MSG\_\-READ}\ 6
\item 
\#define \hyperlink{mainloop__good_8c_a14}{MSG\_\-WRITE}\ 7
\item 
\#define \hyperlink{mainloop__good_8c_a15}{MSG\_\-FULL}\ 8
\item 
\#define \hyperlink{mainloop__good_8c_a16}{MSG\_\-EMPTY}\ 9
\item 
\#define \hyperlink{mainloop__good_8c_a17}{MIN}(a, b)\ ((a)$<$(b)?(a):(b))
\item 
\#define \hyperlink{mainloop__good_8c_a18}{MAX}(a, b)\ ((a)$>$(b)?(a):(b))
\end{CompactItemize}
\subsection*{Typedefs}
\begin{CompactItemize}
\item 
typedef void($\ast$ \hyperlink{mainloop__good_8c_a20}{eventhandler\_\-t} )(int i)
\begin{CompactList}\small\item\em event handler \item\end{CompactList}\item 
typedef void($\ast$ \hyperlink{mainloop__good_8c_a21}{alarmhandler\_\-t} )(void)
\begin{CompactList}\small\item\em alarm handler \item\end{CompactList}\end{CompactItemize}
\subsection*{Functions}
\begin{CompactItemize}
\item 
void \hyperlink{mainloop__good_8c_a26}{message} (int msg)
\begin{CompactList}\small\item\em print message describing current activity \item\end{CompactList}\item 
void \hyperlink{mainloop__good_8c_a27}{blocksigpipe} (void)
\begin{CompactList}\small\item\em block SIGPIPE \item\end{CompactList}\item 
void \hyperlink{mainloop__good_8c_a28}{initalarms} (void)
\begin{CompactList}\small\item\em install non-resetting signal handler for alarms \item\end{CompactList}\item 
void \hyperlink{mainloop__good_8c_a29}{addalarm} (\hyperlink{mainloop__good_8c_a21}{alarmhandler\_\-t} handler, long interval)
\begin{CompactList}\small\item\em add new alarm \item\end{CompactList}\item 
void \hyperlink{mainloop__good_8c_a30}{checkalarms} (void)
\begin{CompactList}\small\item\em check for pending alarms and execute alarm handlers \item\end{CompactList}\item 
void \hyperlink{mainloop__good_8c_a31}{alarmhandler} (void)
\begin{CompactList}\small\item\em signal handler for alarms \item\end{CompactList}\item 
void \hyperlink{mainloop__good_8c_a32}{heartbeat} (void)
\begin{CompactList}\small\item\em print heartbeat message \item\end{CompactList}\item 
void \hyperlink{mainloop__good_8c_a33}{slowheartbeat} (void)
\begin{CompactList}\small\item\em print \char`\"{}slowheartbeat\char`\"{} message \item\end{CompactList}\item 
int \hyperlink{mainloop__good_8c_a34}{listensocket} (int port)
\begin{CompactList}\small\item\em return tcp socket listening on port specified \item\end{CompactList}\item 
void \hyperlink{mainloop__good_8c_a35}{setnonblock} (int fd)
\begin{CompactList}\small\item\em set a file descriptor to be nonblocking \item\end{CompactList}\item 
void \hyperlink{mainloop__good_8c_a36}{initclients} ()
\begin{CompactList}\small\item\em initialize client state array \item\end{CompactList}\item 
void \hyperlink{mainloop__good_8c_a37}{addclient} (int fd, \hyperlink{mainloop__good_8c_a20}{eventhandler\_\-t} read, \hyperlink{mainloop__good_8c_a20}{eventhandler\_\-t} write, \hyperlink{mainloop__good_8c_a20}{eventhandler\_\-t} except)
\begin{CompactList}\small\item\em add new client \item\end{CompactList}\item 
void \hyperlink{mainloop__good_8c_a38}{delclient} (int i)
\begin{CompactList}\small\item\em delete client \item\end{CompactList}\item 
void \hyperlink{mainloop__good_8c_a39}{mainloop} ()
\begin{CompactList}\small\item\em execute one iteration of THE mainloop \item\end{CompactList}\item 
void \hyperlink{mainloop__good_8c_a40}{closeclient} (int i)
\begin{CompactList}\small\item\em close and delete client session \item\end{CompactList}\item 
void \hyperlink{mainloop__good_8c_a41}{acceptecho} (int i)
\begin{CompactList}\small\item\em accept connection and set up new session for \char`\"{}echo\char`\"{} service \item\end{CompactList}\item 
void \hyperlink{mainloop__good_8c_a42}{readecho} (int i)
\begin{CompactList}\small\item\em read data from echo client \item\end{CompactList}\item 
void \hyperlink{mainloop__good_8c_a43}{writeecho} (int i)
\begin{CompactList}\small\item\em write data to an echo client \item\end{CompactList}\item 
void \hyperlink{mainloop__good_8c_a44}{flowecho} (int i)
\begin{CompactList}\small\item\em do \char`\"{}flow\char`\"{} control for echo \item\end{CompactList}\item 
void \hyperlink{mainloop__good_8c_a45}{acceptchargen} (int i)
\begin{CompactList}\small\item\em accept connection and set up new session for \char`\"{}chargen\char`\"{} service \item\end{CompactList}\item 
void \hyperlink{mainloop__good_8c_a46}{writechargen} (int i)
\begin{CompactList}\small\item\em write data to an chargen client \item\end{CompactList}\item 
int \hyperlink{mainloop__good_8c_a47}{main} ()
\end{CompactItemize}
\subsection*{Variables}
\begin{CompactItemize}
\item 
char \hyperlink{mainloop__good_8c_a19}{chargen\_\-buf} \mbox{[}$\,$\mbox{]} = \char`\"{}0123456789abcdefghijklmnopqrstuv\char`\"{}
\begin{CompactList}\small\item\em characters to return in chargen service \item\end{CompactList}\item 
unsigned long \hyperlink{mainloop__good_8c_a22}{npollfd}
\begin{CompactList}\small\item\em number of file descriptors to check \item\end{CompactList}\item 
\hyperlink{structclient__t}{client\_\-t} \hyperlink{mainloop__good_8c_a23}{clients} \mbox{[}MAXCLIENTS\mbox{]}
\begin{CompactList}\small\item\em array of client states \item\end{CompactList}\item 
pollfd \hyperlink{mainloop__good_8c_a24}{pollfds} \mbox{[}MAXCLIENTS\mbox{]}
\begin{CompactList}\small\item\em pollfds for poll() \item\end{CompactList}\item 
\hyperlink{structalarm__t}{alarm\_\-t} \hyperlink{mainloop__good_8c_a25}{alarms} \mbox{[}MAXALARMS\mbox{]}
\begin{CompactList}\small\item\em alarms \item\end{CompactList}\end{CompactItemize}


\subsection{Detailed Description}
Example for \char`\"{}good\char`\"{} main loop. 

\begin{Desc}
\item[Author:]Rico Pajarola\end{Desc}
This example does essentially what glib would do: events are abstracted using callbacks making the mainloop generic (but not as generic as glib). Even though it is easy to add a new input or event source, the whole mechanism is still very much tied to the structure of this program).

Definition in file \hyperlink{mainloop__good_8c-source}{mainloop\_\-good.c}.

\subsection{Define Documentation}
\hypertarget{mainloop__good_8c_a4}{
\index{mainloop_good.c@{mainloop\_\-good.c}!BUFSIZE@{BUFSIZE}}
\index{BUFSIZE@{BUFSIZE}!mainloop_good.c@{mainloop\_\-good.c}}
\subsubsection[BUFSIZE]{\setlength{\rightskip}{0pt plus 5cm}\#define BUFSIZE\ 16}}
\label{mainloop__good_8c_a4}


buffer size for echo client 



Definition at line 37 of file mainloop\_\-good.c.\hypertarget{mainloop__good_8c_a2}{
\index{mainloop_good.c@{mainloop\_\-good.c}!HEARTBEAT_INTERVAL@{HEARTBEAT\_\-INTERVAL}}
\index{HEARTBEAT_INTERVAL@{HEARTBEAT\_\-INTERVAL}!mainloop_good.c@{mainloop\_\-good.c}}
\subsubsection[HEARTBEAT\_\-INTERVAL]{\setlength{\rightskip}{0pt plus 5cm}\#define HEARTBEAT\_\-INTERVAL\ 2}}
\label{mainloop__good_8c_a2}


heartbeat interval (in seconds) 



Definition at line 31 of file mainloop\_\-good.c.\hypertarget{mainloop__good_8c_a18}{
\index{mainloop_good.c@{mainloop\_\-good.c}!MAX@{MAX}}
\index{MAX@{MAX}!mainloop_good.c@{mainloop\_\-good.c}}
\subsubsection[MAX]{\setlength{\rightskip}{0pt plus 5cm}\#define MAX(a, b)\ ((a)$>$(b)?(a):(b))}}
\label{mainloop__good_8c_a18}




Definition at line 57 of file mainloop\_\-good.c.\hypertarget{mainloop__good_8c_a6}{
\index{mainloop_good.c@{mainloop\_\-good.c}!MAXALARMS@{MAXALARMS}}
\index{MAXALARMS@{MAXALARMS}!mainloop_good.c@{mainloop\_\-good.c}}
\subsubsection[MAXALARMS]{\setlength{\rightskip}{0pt plus 5cm}\#define MAXALARMS\ 8}}
\label{mainloop__good_8c_a6}


maximum number of different alarms 



Definition at line 43 of file mainloop\_\-good.c.

Referenced by addalarm(), alarmhandler(), checkalarms(), and initalarms().\hypertarget{mainloop__good_8c_a5}{
\index{mainloop_good.c@{mainloop\_\-good.c}!MAXCLIENTS@{MAXCLIENTS}}
\index{MAXCLIENTS@{MAXCLIENTS}!mainloop_good.c@{mainloop\_\-good.c}}
\subsubsection[MAXCLIENTS]{\setlength{\rightskip}{0pt plus 5cm}\#define MAXCLIENTS\ 8}}
\label{mainloop__good_8c_a5}


maximum number of clients 



Definition at line 40 of file mainloop\_\-good.c.\hypertarget{mainloop__good_8c_a17}{
\index{mainloop_good.c@{mainloop\_\-good.c}!MIN@{MIN}}
\index{MIN@{MIN}!mainloop_good.c@{mainloop\_\-good.c}}
\subsubsection[MIN]{\setlength{\rightskip}{0pt plus 5cm}\#define MIN(a, b)\ ((a)$<$(b)?(a):(b))}}
\label{mainloop__good_8c_a17}




Definition at line 56 of file mainloop\_\-good.c.\hypertarget{mainloop__good_8c_a10}{
\index{mainloop_good.c@{mainloop\_\-good.c}!MSG_ACCEPT@{MSG\_\-ACCEPT}}
\index{MSG_ACCEPT@{MSG\_\-ACCEPT}!mainloop_good.c@{mainloop\_\-good.c}}
\subsubsection[MSG\_\-ACCEPT]{\setlength{\rightskip}{0pt plus 5cm}\#define MSG\_\-ACCEPT\ 3}}
\label{mainloop__good_8c_a10}




Definition at line 48 of file mainloop\_\-good.c.\hypertarget{mainloop__good_8c_a12}{
\index{mainloop_good.c@{mainloop\_\-good.c}!MSG_CLOSE@{MSG\_\-CLOSE}}
\index{MSG_CLOSE@{MSG\_\-CLOSE}!mainloop_good.c@{mainloop\_\-good.c}}
\subsubsection[MSG\_\-CLOSE]{\setlength{\rightskip}{0pt plus 5cm}\#define MSG\_\-CLOSE\ 5}}
\label{mainloop__good_8c_a12}




Definition at line 50 of file mainloop\_\-good.c.\hypertarget{mainloop__good_8c_a16}{
\index{mainloop_good.c@{mainloop\_\-good.c}!MSG_EMPTY@{MSG\_\-EMPTY}}
\index{MSG_EMPTY@{MSG\_\-EMPTY}!mainloop_good.c@{mainloop\_\-good.c}}
\subsubsection[MSG\_\-EMPTY]{\setlength{\rightskip}{0pt plus 5cm}\#define MSG\_\-EMPTY\ 9}}
\label{mainloop__good_8c_a16}




Definition at line 54 of file mainloop\_\-good.c.\hypertarget{mainloop__good_8c_a15}{
\index{mainloop_good.c@{mainloop\_\-good.c}!MSG_FULL@{MSG\_\-FULL}}
\index{MSG_FULL@{MSG\_\-FULL}!mainloop_good.c@{mainloop\_\-good.c}}
\subsubsection[MSG\_\-FULL]{\setlength{\rightskip}{0pt plus 5cm}\#define MSG\_\-FULL\ 8}}
\label{mainloop__good_8c_a15}




Definition at line 53 of file mainloop\_\-good.c.\hypertarget{mainloop__good_8c_a7}{
\index{mainloop_good.c@{mainloop\_\-good.c}!MSG_HEARTBEAT@{MSG\_\-HEARTBEAT}}
\index{MSG_HEARTBEAT@{MSG\_\-HEARTBEAT}!mainloop_good.c@{mainloop\_\-good.c}}
\subsubsection[MSG\_\-HEARTBEAT]{\setlength{\rightskip}{0pt plus 5cm}\#define MSG\_\-HEARTBEAT\ 0}}
\label{mainloop__good_8c_a7}




Definition at line 45 of file mainloop\_\-good.c.\hypertarget{mainloop__good_8c_a9}{
\index{mainloop_good.c@{mainloop\_\-good.c}!MSG_MAINLOOP@{MSG\_\-MAINLOOP}}
\index{MSG_MAINLOOP@{MSG\_\-MAINLOOP}!mainloop_good.c@{mainloop\_\-good.c}}
\subsubsection[MSG\_\-MAINLOOP]{\setlength{\rightskip}{0pt plus 5cm}\#define MSG\_\-MAINLOOP\ 2}}
\label{mainloop__good_8c_a9}




Definition at line 47 of file mainloop\_\-good.c.\hypertarget{mainloop__good_8c_a13}{
\index{mainloop_good.c@{mainloop\_\-good.c}!MSG_READ@{MSG\_\-READ}}
\index{MSG_READ@{MSG\_\-READ}!mainloop_good.c@{mainloop\_\-good.c}}
\subsubsection[MSG\_\-READ]{\setlength{\rightskip}{0pt plus 5cm}\#define MSG\_\-READ\ 6}}
\label{mainloop__good_8c_a13}




Definition at line 51 of file mainloop\_\-good.c.\hypertarget{mainloop__good_8c_a8}{
\index{mainloop_good.c@{mainloop\_\-good.c}!MSG_SLOWHEARTBEAT@{MSG\_\-SLOWHEARTBEAT}}
\index{MSG_SLOWHEARTBEAT@{MSG\_\-SLOWHEARTBEAT}!mainloop_good.c@{mainloop\_\-good.c}}
\subsubsection[MSG\_\-SLOWHEARTBEAT]{\setlength{\rightskip}{0pt plus 5cm}\#define MSG\_\-SLOWHEARTBEAT\ 1}}
\label{mainloop__good_8c_a8}




Definition at line 46 of file mainloop\_\-good.c.\hypertarget{mainloop__good_8c_a11}{
\index{mainloop_good.c@{mainloop\_\-good.c}!MSG_TOOMANY@{MSG\_\-TOOMANY}}
\index{MSG_TOOMANY@{MSG\_\-TOOMANY}!mainloop_good.c@{mainloop\_\-good.c}}
\subsubsection[MSG\_\-TOOMANY]{\setlength{\rightskip}{0pt plus 5cm}\#define MSG\_\-TOOMANY\ 4}}
\label{mainloop__good_8c_a11}




Definition at line 49 of file mainloop\_\-good.c.\hypertarget{mainloop__good_8c_a14}{
\index{mainloop_good.c@{mainloop\_\-good.c}!MSG_WRITE@{MSG\_\-WRITE}}
\index{MSG_WRITE@{MSG\_\-WRITE}!mainloop_good.c@{mainloop\_\-good.c}}
\subsubsection[MSG\_\-WRITE]{\setlength{\rightskip}{0pt plus 5cm}\#define MSG\_\-WRITE\ 7}}
\label{mainloop__good_8c_a14}




Definition at line 52 of file mainloop\_\-good.c.\hypertarget{mainloop__good_8c_a1}{
\index{mainloop_good.c@{mainloop\_\-good.c}!PORT_CHARGEN@{PORT\_\-CHARGEN}}
\index{PORT_CHARGEN@{PORT\_\-CHARGEN}!mainloop_good.c@{mainloop\_\-good.c}}
\subsubsection[PORT\_\-CHARGEN]{\setlength{\rightskip}{0pt plus 5cm}\#define PORT\_\-CHARGEN\ 5006}}
\label{mainloop__good_8c_a1}


tcp port to listen for chargen clients 



Definition at line 28 of file mainloop\_\-good.c.\hypertarget{mainloop__good_8c_a0}{
\index{mainloop_good.c@{mainloop\_\-good.c}!PORT_ECHO@{PORT\_\-ECHO}}
\index{PORT_ECHO@{PORT\_\-ECHO}!mainloop_good.c@{mainloop\_\-good.c}}
\subsubsection[PORT\_\-ECHO]{\setlength{\rightskip}{0pt plus 5cm}\#define PORT\_\-ECHO\ 5005}}
\label{mainloop__good_8c_a0}


tcp port to listen for echo clients 



Definition at line 25 of file mainloop\_\-good.c.\hypertarget{mainloop__good_8c_a3}{
\index{mainloop_good.c@{mainloop\_\-good.c}!SLOWHEARTBEAT_INTERVAL@{SLOWHEARTBEAT\_\-INTERVAL}}
\index{SLOWHEARTBEAT_INTERVAL@{SLOWHEARTBEAT\_\-INTERVAL}!mainloop_good.c@{mainloop\_\-good.c}}
\subsubsection[SLOWHEARTBEAT\_\-INTERVAL]{\setlength{\rightskip}{0pt plus 5cm}\#define SLOWHEARTBEAT\_\-INTERVAL\ 15}}
\label{mainloop__good_8c_a3}


slow heartbeat interval (in seconds) 



Definition at line 34 of file mainloop\_\-good.c.

\subsection{Typedef Documentation}
\hypertarget{mainloop__good_8c_a21}{
\index{mainloop_good.c@{mainloop\_\-good.c}!alarmhandler_t@{alarmhandler\_\-t}}
\index{alarmhandler_t@{alarmhandler\_\-t}!mainloop_good.c@{mainloop\_\-good.c}}
\subsubsection[alarmhandler\_\-t]{\setlength{\rightskip}{0pt plus 5cm}typedef void($\ast$ \hyperlink{mainloop__good_8c_a21}{alarmhandler\_\-t})(void)}}
\label{mainloop__good_8c_a21}


alarm handler 



Definition at line 106 of file mainloop\_\-good.c.\hypertarget{mainloop__good_8c_a20}{
\index{mainloop_good.c@{mainloop\_\-good.c}!eventhandler_t@{eventhandler\_\-t}}
\index{eventhandler_t@{eventhandler\_\-t}!mainloop_good.c@{mainloop\_\-good.c}}
\subsubsection[eventhandler\_\-t]{\setlength{\rightskip}{0pt plus 5cm}typedef void($\ast$ \hyperlink{mainloop__good_8c_a20}{eventhandler\_\-t})(int i)}}
\label{mainloop__good_8c_a20}


event handler 

\begin{Desc}
\item[Parameters:]
\begin{description}
\item[{\em i}]index into clients \end{description}
\end{Desc}


Definition at line 84 of file mainloop\_\-good.c.

\subsection{Function Documentation}
\hypertarget{mainloop__good_8c_a45}{
\index{mainloop_good.c@{mainloop\_\-good.c}!acceptchargen@{acceptchargen}}
\index{acceptchargen@{acceptchargen}!mainloop_good.c@{mainloop\_\-good.c}}
\subsubsection[acceptchargen]{\setlength{\rightskip}{0pt plus 5cm}void acceptchargen (int {\em i})\hspace{0.3cm}{\tt  \mbox{[}static\mbox{]}}}}
\label{mainloop__good_8c_a45}


accept connection and set up new session for \char`\"{}chargen\char`\"{} service 

\begin{Desc}
\item[Parameters:]
\begin{description}
\item[{\em i}]index into pollfds/clients array for server descriptor \end{description}
\end{Desc}


Definition at line 680 of file mainloop\_\-good.c.

References addclient(), clients, closeclient(), and writechargen().

Referenced by main().



\footnotesize\begin{verbatim}681 {
682     int             fd;
683     socklen_t       t;
684     struct sockaddr_in sain;
685 
686     t = sizeof(sain);
687     if ((fd = accept(clients[i].fd, (void *) &sain, &t)) == -1) {
688         if (errno == EWOULDBLOCK) {
689             return;
690         }
691         perror("accept(CHARGEN)");
692         exit(EXIT_FAILURE);
693     }
694     addclient(fd, NULL, &writechargen, &closeclient);
695 }
\end{verbatim}\normalsize 
\hypertarget{mainloop__good_8c_a41}{
\index{mainloop_good.c@{mainloop\_\-good.c}!acceptecho@{acceptecho}}
\index{acceptecho@{acceptecho}!mainloop_good.c@{mainloop\_\-good.c}}
\subsubsection[acceptecho]{\setlength{\rightskip}{0pt plus 5cm}void acceptecho (int {\em i})\hspace{0.3cm}{\tt  \mbox{[}static\mbox{]}}}}
\label{mainloop__good_8c_a41}


accept connection and set up new session for \char`\"{}echo\char`\"{} service 

\begin{Desc}
\item[Parameters:]
\begin{description}
\item[{\em i}]index into pollfds/clients array for server descriptor \end{description}
\end{Desc}


Definition at line 524 of file mainloop\_\-good.c.

References addclient(), clients, closeclient(), message(), MSG\_\-ACCEPT, readecho(), and writeecho().

Referenced by main().



\footnotesize\begin{verbatim}525 {
526     int             fd;
527     socklen_t       t;
528     struct sockaddr_in sain;
529 
530     t = sizeof(sain);
531     if ((fd = accept(clients[i].fd, (void *) &sain, &t)) == -1) {
532         if (errno == EWOULDBLOCK) {
533             return;
534         }
535         perror("accept(ECHO)");
536         exit(EXIT_FAILURE);
537     }
538     message(MSG_ACCEPT);
539     addclient(fd, &readecho, &writeecho, &closeclient);
540 }
\end{verbatim}\normalsize 
\hypertarget{mainloop__good_8c_a29}{
\index{mainloop_good.c@{mainloop\_\-good.c}!addalarm@{addalarm}}
\index{addalarm@{addalarm}!mainloop_good.c@{mainloop\_\-good.c}}
\subsubsection[addalarm]{\setlength{\rightskip}{0pt plus 5cm}void addalarm (\hyperlink{mainloop__good_8c_a21}{alarmhandler\_\-t} {\em handler}, long {\em interval})\hspace{0.3cm}{\tt  \mbox{[}static\mbox{]}}}}
\label{mainloop__good_8c_a29}


add new alarm 

\begin{Desc}
\item[Parameters:]
\begin{description}
\item[{\em handler}]alarm handler procedure \item[{\em interval}]interval in seconds\end{description}
\end{Desc}


\begin{Desc}
\item[\hyperlink{bug__bug000001}{Bug}]no error handling, if there are no more error handler slots, the new alarm is ignored... \end{Desc}


Definition at line 263 of file mainloop\_\-good.c.

References alarmhandler(), alarms, alarm\_\-t::flag, alarm\_\-t::handler, alarm\_\-t::interval, MAXALARMS, and alarm\_\-t::nexttime.

Referenced by main().



\footnotesize\begin{verbatim}264 {
265     struct timeval  now;
266     int             i;
267 
268     for (i = 0; i < MAXALARMS; i++) {
269         if (alarms[i].handler == NULL) {
270             gettimeofday(&now, NULL);
271             alarms[i].handler = handler;
272             alarms[i].interval = interval;
273             alarms[i].nexttime = now.tv_sec + interval;
274             alarms[i].flag = 0;
275             alarmhandler();
276             return;
277         }
278     }
279 }
\end{verbatim}\normalsize 
\hypertarget{mainloop__good_8c_a37}{
\index{mainloop_good.c@{mainloop\_\-good.c}!addclient@{addclient}}
\index{addclient@{addclient}!mainloop_good.c@{mainloop\_\-good.c}}
\subsubsection[addclient]{\setlength{\rightskip}{0pt plus 5cm}void addclient (int {\em fd}, \hyperlink{mainloop__good_8c_a20}{eventhandler\_\-t} {\em read}, \hyperlink{mainloop__good_8c_a20}{eventhandler\_\-t} {\em write}, \hyperlink{mainloop__good_8c_a20}{eventhandler\_\-t} {\em except})\hspace{0.3cm}{\tt  \mbox{[}static\mbox{]}}}}
\label{mainloop__good_8c_a37}


add new client 

\begin{Desc}
\item[Parameters:]
\begin{description}
\item[{\em fd}]file descriptor asociated with this client \item[{\em read}]handler called if socket is readable \item[{\em write}]handler called if socket is writable \item[{\em except}]handler called on exceptions (HUP, close etc). \end{description}
\end{Desc}


Definition at line 437 of file mainloop\_\-good.c.

References clients, client\_\-t::except, client\_\-t::fd, MAXCLIENTS, message(), MSG\_\-TOOMANY, npollfd, pollfds, client\_\-t::read, and client\_\-t::write.

Referenced by acceptchargen(), acceptecho(), and main().



\footnotesize\begin{verbatim}439 {
440     if (npollfd >= (MAXCLIENTS - 1)) {
441         close(fd);
442         message(MSG_TOOMANY);
443         return;
444     }
445     clients[npollfd].fd = fd;
446     clients[npollfd].read = read;
447     clients[npollfd].write = write;
448     clients[npollfd].except = except;
449     pollfds[npollfd].fd = fd;
450     pollfds[npollfd].events = 0;
451     if (read) {
452         pollfds[npollfd].events |= POLLIN;
453     }
454     if (write) {
455         pollfds[npollfd].events |= POLLOUT;
456     }
457     npollfd++;
458 }
\end{verbatim}\normalsize 
\hypertarget{mainloop__good_8c_a31}{
\index{mainloop_good.c@{mainloop\_\-good.c}!alarmhandler@{alarmhandler}}
\index{alarmhandler@{alarmhandler}!mainloop_good.c@{mainloop\_\-good.c}}
\subsubsection[alarmhandler]{\setlength{\rightskip}{0pt plus 5cm}void alarmhandler (void)\hspace{0.3cm}{\tt  \mbox{[}static\mbox{]}}}}
\label{mainloop__good_8c_a31}


signal handler for alarms 

mark expired alarms for execution and set new alarm timer.

the resolution of the alarm timer is one second, no attempt is made to get timing beyond this one second resolution (any sub-second timing information is discarded). 

Definition at line 307 of file mainloop\_\-good.c.

References alarms, alarm\_\-t::flag, alarm\_\-t::handler, alarm\_\-t::interval, MAX, MAXALARMS, MIN, and alarm\_\-t::nexttime.

Referenced by addalarm(), and initalarms().



\footnotesize\begin{verbatim}308 {
309     struct timeval  now;
310     long            nextalarm;
311     int             i;
312 
313     gettimeofday(&now, NULL);
314 
315     nextalarm = 65536;
316     for (i = 0; i < MAXALARMS; i++) {
317         if (alarms[i].handler != NULL) {
318             if (alarms[i].nexttime <= now.tv_sec) {
319                 alarms[i].flag = 1;
320                 alarms[i].nexttime += alarms[i].interval;
321             }
322             nextalarm =
323                 MIN(nextalarm,
324                     MAX(1,
325                         (unsigned int) alarms[i].nexttime - now.tv_sec));
326         }
327     }
328     alarm((unsigned int) nextalarm);
329 }
\end{verbatim}\normalsize 
\hypertarget{mainloop__good_8c_a27}{
\index{mainloop_good.c@{mainloop\_\-good.c}!blocksigpipe@{blocksigpipe}}
\index{blocksigpipe@{blocksigpipe}!mainloop_good.c@{mainloop\_\-good.c}}
\subsubsection[blocksigpipe]{\setlength{\rightskip}{0pt plus 5cm}void blocksigpipe (void)\hspace{0.3cm}{\tt  \mbox{[}static\mbox{]}}}}
\label{mainloop__good_8c_a27}


block SIGPIPE 

Trying to write to a socket when the other end has already closed the connection results in SIGPIPE. Not usefull in this context. 

Definition at line 214 of file mainloop\_\-good.c.



\footnotesize\begin{verbatim}215 {
216     struct sigaction act;
217 
218     act.sa_handler = SIG_IGN;
219     sigemptyset(&act.sa_mask);
220     act.sa_flags = SA_RESTART;
221     if (sigaction(SIGPIPE, &act, NULL) == -1) {
222         perror("sigaction(SIGPIPE, <ignore>)");
223         exit(EXIT_FAILURE);
224     }
225 }
\end{verbatim}\normalsize 
\hypertarget{mainloop__good_8c_a30}{
\index{mainloop_good.c@{mainloop\_\-good.c}!checkalarms@{checkalarms}}
\index{checkalarms@{checkalarms}!mainloop_good.c@{mainloop\_\-good.c}}
\subsubsection[checkalarms]{\setlength{\rightskip}{0pt plus 5cm}void checkalarms (void)\hspace{0.3cm}{\tt  \mbox{[}static\mbox{]}}}}
\label{mainloop__good_8c_a30}


check for pending alarms and execute alarm handlers 



Definition at line 285 of file mainloop\_\-good.c.

References alarms, alarm\_\-t::flag, alarm\_\-t::handler, and MAXALARMS.

Referenced by mainloop().



\footnotesize\begin{verbatim}286 {
287     int             i;
288 
289     for (i = 0; i < MAXALARMS; i++) {
290         if (alarms[i].flag) {
291             (*alarms[i].handler) ();
292             alarms[i].flag = 0;
293         }
294     }
295 }
\end{verbatim}\normalsize 
\hypertarget{mainloop__good_8c_a40}{
\index{mainloop_good.c@{mainloop\_\-good.c}!closeclient@{closeclient}}
\index{closeclient@{closeclient}!mainloop_good.c@{mainloop\_\-good.c}}
\subsubsection[closeclient]{\setlength{\rightskip}{0pt plus 5cm}void closeclient (int {\em i})\hspace{0.3cm}{\tt  \mbox{[}static\mbox{]}}}}
\label{mainloop__good_8c_a40}


close and delete client session 

\begin{Desc}
\item[Parameters:]
\begin{description}
\item[{\em i}]index into pollfds/clients array \end{description}
\end{Desc}


Definition at line 666 of file mainloop\_\-good.c.

References clients, delclient(), client\_\-t::fd, message(), and MSG\_\-CLOSE.

Referenced by acceptchargen(), acceptecho(), readecho(), writechargen(), and writeecho().



\footnotesize\begin{verbatim}667 {
668     message(MSG_CLOSE);
669     close(clients[i].fd);
670     delclient(i);
671     clients[i].fd = -1;
672 }
\end{verbatim}\normalsize 
\hypertarget{mainloop__good_8c_a38}{
\index{mainloop_good.c@{mainloop\_\-good.c}!delclient@{delclient}}
\index{delclient@{delclient}!mainloop_good.c@{mainloop\_\-good.c}}
\subsubsection[delclient]{\setlength{\rightskip}{0pt plus 5cm}void delclient (int {\em i})\hspace{0.3cm}{\tt  \mbox{[}static\mbox{]}}}}
\label{mainloop__good_8c_a38}


delete client 

\begin{Desc}
\item[Parameters:]
\begin{description}
\item[{\em i}]index into pollfds/clients array \end{description}
\end{Desc}


Definition at line 466 of file mainloop\_\-good.c.

References clients, client\_\-t::fd, npollfd, and pollfds.

Referenced by closeclient().



\footnotesize\begin{verbatim}467 {
468     clients[i].fd = -1;
469     if (i != npollfd) {
470         /* just copy the whole thing including the buffer... */
471         memcpy(&clients[i], &clients[npollfd], sizeof(clients[i]));
472         pollfds[i] = pollfds[npollfd];
473         npollfd--;
474     }
475 }
\end{verbatim}\normalsize 
\hypertarget{mainloop__good_8c_a44}{
\index{mainloop_good.c@{mainloop\_\-good.c}!flowecho@{flowecho}}
\index{flowecho@{flowecho}!mainloop_good.c@{mainloop\_\-good.c}}
\subsubsection[flowecho]{\setlength{\rightskip}{0pt plus 5cm}void flowecho (int {\em i})\hspace{0.3cm}{\tt  \mbox{[}static\mbox{]}}}}
\label{mainloop__good_8c_a44}


do \char`\"{}flow\char`\"{} control for echo 

\begin{Desc}
\item[Parameters:]
\begin{description}
\item[{\em i}]index into pollfds/clients array for echo session\end{description}
\end{Desc}
turn off checking for read if buffer is full resp. write if buffer is empty. 

Definition at line 646 of file mainloop\_\-good.c.

References BUFSIZE, clients, client\_\-t::echo, echo\_\-client\_\-t::n, and pollfds.

Referenced by readecho(), and writeecho().



\footnotesize\begin{verbatim}647 {
648     switch (clients[i].echo.n) {
649     case 0:
650         pollfds[i].events = POLLIN;
651         break;
652     case BUFSIZE:
653         pollfds[i].events = POLLOUT;
654         break;
655     default:
656         pollfds[i].events = POLLIN | POLLOUT;
657     }
658 }
\end{verbatim}\normalsize 
\hypertarget{mainloop__good_8c_a32}{
\index{mainloop_good.c@{mainloop\_\-good.c}!heartbeat@{heartbeat}}
\index{heartbeat@{heartbeat}!mainloop_good.c@{mainloop\_\-good.c}}
\subsubsection[heartbeat]{\setlength{\rightskip}{0pt plus 5cm}void heartbeat (void)\hspace{0.3cm}{\tt  \mbox{[}static\mbox{]}}}}
\label{mainloop__good_8c_a32}


print heartbeat message 



Definition at line 335 of file mainloop\_\-good.c.

References message(), and MSG\_\-HEARTBEAT.

Referenced by main().



\footnotesize\begin{verbatim}336 {
337     message(MSG_HEARTBEAT);
338 }
\end{verbatim}\normalsize 
\hypertarget{mainloop__good_8c_a28}{
\index{mainloop_good.c@{mainloop\_\-good.c}!initalarms@{initalarms}}
\index{initalarms@{initalarms}!mainloop_good.c@{mainloop\_\-good.c}}
\subsubsection[initalarms]{\setlength{\rightskip}{0pt plus 5cm}void initalarms (void)\hspace{0.3cm}{\tt  \mbox{[}static\mbox{]}}}}
\label{mainloop__good_8c_a28}


install non-resetting signal handler for alarms 

The sigaction interface allows installing non-resetting signal handlers (ie not reset to SIG\_\-DFL after disposition). This way, there is no race condition when reinstalling the signal handler. 

Definition at line 235 of file mainloop\_\-good.c.

References alarmhandler(), alarms, alarm\_\-t::handler, and MAXALARMS.

Referenced by main().



\footnotesize\begin{verbatim}236 {
237     struct sigaction act;
238     int             i;
239 
240     act.sa_handler = alarmhandler;
241     sigemptyset(&act.sa_mask);
242     act.sa_flags = 0;           /* no SIG_RESTART! */
243     if (sigaction(SIGALRM, &act, NULL) == -1) {
244         perror("sigaction(SIGPIPE, <ignore>)");
245         exit(EXIT_FAILURE);
246     }
247     for (i = 0; i < MAXALARMS; i++) {
248         alarms[i].handler = NULL;
249     }
250     alarm(0);                   /* cancel any previously made alarm request */
251 }
\end{verbatim}\normalsize 
\hypertarget{mainloop__good_8c_a36}{
\index{mainloop_good.c@{mainloop\_\-good.c}!initclients@{initclients}}
\index{initclients@{initclients}!mainloop_good.c@{mainloop\_\-good.c}}
\subsubsection[initclients]{\setlength{\rightskip}{0pt plus 5cm}void initclients ()\hspace{0.3cm}{\tt  \mbox{[}static\mbox{]}}}}
\label{mainloop__good_8c_a36}


initialize client state array 



Definition at line 418 of file mainloop\_\-good.c.

References clients, client\_\-t::fd, MAXCLIENTS, and npollfd.

Referenced by main().



\footnotesize\begin{verbatim}419 {
420     int             i;
421 
422     for (i = 0; i < MAXCLIENTS; i++) {
423         clients[i].fd = -1;
424     }
425     npollfd = 0;
426 }
\end{verbatim}\normalsize 
\hypertarget{mainloop__good_8c_a34}{
\index{mainloop_good.c@{mainloop\_\-good.c}!listensocket@{listensocket}}
\index{listensocket@{listensocket}!mainloop_good.c@{mainloop\_\-good.c}}
\subsubsection[listensocket]{\setlength{\rightskip}{0pt plus 5cm}int listensocket (int {\em port})\hspace{0.3cm}{\tt  \mbox{[}static\mbox{]}}}}
\label{mainloop__good_8c_a34}


return tcp socket listening on port specified 

\begin{Desc}
\item[Parameters:]
\begin{description}
\item[{\em port}]port number in host byte order\end{description}
\end{Desc}
\begin{Desc}
\item[Returns:]file descriptor for listening socket\end{Desc}
The socket is set to be nonblocking 

Definition at line 359 of file mainloop\_\-good.c.

References setnonblock().

Referenced by listensource(), and main().



\footnotesize\begin{verbatim}360 {
361     int             serverfd;
362     struct sockaddr_in sain;
363     int             one;
364 
365     /* set up tcp socket for listening */
366     sain.sin_family = AF_INET;
367     sain.sin_port = htons(port);
368     sain.sin_addr.s_addr = INADDR_ANY;
369     if ((serverfd = socket(AF_INET, SOCK_STREAM, 0)) == -1) {
370         perror("socket(AF_INET, SOCK_STREAM, 0)");
371         exit(EXIT_FAILURE);
372     }
373     one = 1;
374     if (setsockopt
375         (serverfd, SOL_SOCKET, SO_REUSEADDR, &one,
376          (int) sizeof(one)) == -1) {
377         perror("setsockopt(SO_REUSEADDR)");
378         exit(EXIT_FAILURE);
379     }
380     if (bind
381         (serverfd, (struct sockaddr *) &sain, sizeof(struct sockaddr_in))
382         == -1) {
383         perror("bind()");
384         exit(EXIT_FAILURE);
385     }
386     if (listen(serverfd, 5) == -1) {
387         perror("listen()");
388         exit(EXIT_FAILURE);
389     }
390     setnonblock(serverfd);
391 
392     printf("listening on port %d\n", port);
393 
394     return serverfd;
395 }
\end{verbatim}\normalsize 
\hypertarget{mainloop__good_8c_a47}{
\index{mainloop_good.c@{mainloop\_\-good.c}!main@{main}}
\index{main@{main}!mainloop_good.c@{mainloop\_\-good.c}}
\subsubsection[main]{\setlength{\rightskip}{0pt plus 5cm}int main ()}}
\label{mainloop__good_8c_a47}




Definition at line 734 of file mainloop\_\-good.c.

References acceptchargen(), acceptecho(), addalarm(), addclient(), blocksigpipe(), heartbeat(), HEARTBEAT\_\-INTERVAL, initalarms(), initclients(), listensocket(), mainloop(), PORT\_\-CHARGEN, PORT\_\-ECHO, slowheartbeat(), and SLOWHEARTBEAT\_\-INTERVAL.



\footnotesize\begin{verbatim}735 {
736     printf("example: better main loop\n");
737 
738     blocksigpipe();
739 
740     initclients();
741     addclient(listensocket(PORT_ECHO), &acceptecho, NULL, NULL);
742     addclient(listensocket(PORT_CHARGEN), &acceptchargen, NULL, NULL);
743     initalarms();
744     addalarm(heartbeat, HEARTBEAT_INTERVAL);
745     addalarm(slowheartbeat, SLOWHEARTBEAT_INTERVAL);
746     printf("heartbeat every %d seconds\n", HEARTBEAT_INTERVAL);
747     printf("slow heartbeat every %d seconds\n", SLOWHEARTBEAT_INTERVAL);
748 
749     /* main loop */
750     while (1) {
751         mainloop();
752     }
753     /* notreached */
754 }
\end{verbatim}\normalsize 
\hypertarget{mainloop__good_8c_a39}{
\index{mainloop_good.c@{mainloop\_\-good.c}!mainloop@{mainloop}}
\index{mainloop@{mainloop}!mainloop_good.c@{mainloop\_\-good.c}}
\subsubsection[mainloop]{\setlength{\rightskip}{0pt plus 5cm}void mainloop ()\hspace{0.3cm}{\tt  \mbox{[}static\mbox{]}}}}
\label{mainloop__good_8c_a39}


execute one iteration of THE mainloop 



Definition at line 481 of file mainloop\_\-good.c.

References checkalarms(), clients, client\_\-t::except, message(), MSG\_\-MAINLOOP, npollfd, pollfds, client\_\-t::read, and client\_\-t::write.

Referenced by main().



\footnotesize\begin{verbatim}482 {
483     int             i;
484 
485     message(MSG_MAINLOOP);
486     i = poll(pollfds, npollfd, -1);
487 
488     /* check for alarm handlers to be executed */
489     checkalarms();
490 
491     if (i == -1) {
492         if ((errno != EAGAIN) && (errno != EINTR)) {
493             perror("poll()");
494             exit(EXIT_FAILURE);
495         }
496         return;
497     }
498 
499     /* 
500      * handle i/o events. work off exceptions first. then read (try to fill
501      * buffer) and write (try to empty buffer)
502      */
503     for (i = 0; i <= npollfd; i++) {
504         if ((clients[i].except)
505             && (pollfds[i].revents & (POLLERR | POLLHUP | POLLNVAL))) {
506             (*clients[i].except) (i);
507             continue;
508         }
509         if ((clients[i].read) && (pollfds[i].revents & POLLIN)) {
510             (*clients[i].read) (i);
511         }
512         if ((clients[i].write) && (pollfds[i].revents & POLLOUT)) {
513             (*clients[i].write) (i);
514         }
515     }
516 }
\end{verbatim}\normalsize 
\hypertarget{mainloop__good_8c_a26}{
\index{mainloop_good.c@{mainloop\_\-good.c}!message@{message}}
\index{message@{message}!mainloop_good.c@{mainloop\_\-good.c}}
\subsubsection[message]{\setlength{\rightskip}{0pt plus 5cm}void message (int {\em msg})\hspace{0.3cm}{\tt  \mbox{[}static\mbox{]}}}}
\label{mainloop__good_8c_a26}


print message describing current activity 

\begin{Desc}
\item[Parameters:]
\begin{description}
\item[{\em msg}]id of message to print (MSG\_\-XYZ) \end{description}
\end{Desc}


Definition at line 170 of file mainloop\_\-good.c.

References MSG\_\-ACCEPT, MSG\_\-CLOSE, MSG\_\-EMPTY, MSG\_\-FULL, MSG\_\-HEARTBEAT, MSG\_\-MAINLOOP, MSG\_\-READ, MSG\_\-SLOWHEARTBEAT, MSG\_\-TOOMANY, and MSG\_\-WRITE.

Referenced by accept\_\-dispatch(), acceptecho(), addclient(), chargen\_\-dispatch(), closeclient(), echo\_\-dispatch\_\-read(), echo\_\-dispatch\_\-write(), heartbeat(), main(), mainloop(), readecho(), sighandler(), slowheartbeat(), source\_\-close(), writechargen(), and writeecho().



\footnotesize\begin{verbatim}171 {
172     switch (msg) {
173     case MSG_HEARTBEAT:
174         printf("H");
175         break;
176     case MSG_SLOWHEARTBEAT:
177         printf("S");
178         break;
179     case MSG_MAINLOOP:
180         /* printf("M"); */
181         break;
182     case MSG_ACCEPT:
183         printf("A");
184         break;
185     case MSG_TOOMANY:
186         printf("T");
187         break;
188     case MSG_CLOSE:
189         printf("C");
190         break;
191     case MSG_READ:
192         printf("R");
193         break;
194     case MSG_WRITE:
195         printf("W");
196         break;
197     case MSG_FULL:
198         printf("F");
199         break;
200     case MSG_EMPTY:
201         printf("E");
202         break;
203     }
204     fflush(stdout);
205 }
\end{verbatim}\normalsize 
\hypertarget{mainloop__good_8c_a42}{
\index{mainloop_good.c@{mainloop\_\-good.c}!readecho@{readecho}}
\index{readecho@{readecho}!mainloop_good.c@{mainloop\_\-good.c}}
\subsubsection[readecho]{\setlength{\rightskip}{0pt plus 5cm}void readecho (int {\em i})\hspace{0.3cm}{\tt  \mbox{[}static\mbox{]}}}}
\label{mainloop__good_8c_a42}


read data from echo client 

\begin{Desc}
\item[Parameters:]
\begin{description}
\item[{\em i}]index into pollfds/clients array for echo session\end{description}
\end{Desc}
If the buffer is not full, tries to do one read from the filedescriptor associated with this echo client. 

Definition at line 551 of file mainloop\_\-good.c.

References BUFSIZE, clients, closeclient(), client\_\-t::echo, flowecho(), message(), MSG\_\-FULL, MSG\_\-READ, echo\_\-client\_\-t::n, echo\_\-client\_\-t::r, and echo\_\-client\_\-t::w.

Referenced by acceptecho(), and main().



\footnotesize\begin{verbatim}552 {
553     ssize_t         nread;
554 
555     if (clients[i].echo.n == BUFSIZE) {
556         message(MSG_FULL);
557         flowecho(i);
558         return;
559     }
560     if (clients[i].echo.r >= clients[i].echo.w) {
561         nread =
562             read(clients[i].fd, clients[i].echo.buf + clients[i].echo.r,
563                  BUFSIZE - clients[i].echo.r);
564     } else {
565         nread =
566             read(clients[i].fd, clients[i].echo.buf + clients[i].echo.r,
567                  clients[i].echo.w - clients[i].echo.r);
568     }
569 
570     switch (nread) {
571     case -1:
572         if ((errno != EINTR) && (errno != EWOULDBLOCK)) {
573             perror("read()");
574             exit(EXIT_FAILURE);
575         }
576         break;
577     case 0:
578         closeclient(i);
579         return;
580     default:
581         message(MSG_READ);
582         clients[i].echo.n += nread;
583         clients[i].echo.r += nread;
584         clients[i].echo.r %= BUFSIZE;
585         flowecho(i);
586     }
587 }
\end{verbatim}\normalsize 
\hypertarget{mainloop__good_8c_a35}{
\index{mainloop_good.c@{mainloop\_\-good.c}!setnonblock@{setnonblock}}
\index{setnonblock@{setnonblock}!mainloop_good.c@{mainloop\_\-good.c}}
\subsubsection[setnonblock]{\setlength{\rightskip}{0pt plus 5cm}void setnonblock (int {\em fd})\hspace{0.3cm}{\tt  \mbox{[}static\mbox{]}}}}
\label{mainloop__good_8c_a35}


set a file descriptor to be nonblocking 

\begin{Desc}
\item[Parameters:]
\begin{description}
\item[{\em fd}]file descriptor\end{description}
\end{Desc}
Non-Blocking works only for Sockets, Pipes and slow devices, it has no effect when used with regular files or \char`\"{}fast\char`\"{} devices. 

Definition at line 406 of file mainloop\_\-good.c.

Referenced by accept\_\-dispatch(), listensocket(), and main().



\footnotesize\begin{verbatim}407 {
408     int             flag;
409 
410     flag = fcntl(fd, F_GETFL);
411     fcntl(fd, F_GETFL, flag | O_NONBLOCK);
412 }
\end{verbatim}\normalsize 
\hypertarget{mainloop__good_8c_a33}{
\index{mainloop_good.c@{mainloop\_\-good.c}!slowheartbeat@{slowheartbeat}}
\index{slowheartbeat@{slowheartbeat}!mainloop_good.c@{mainloop\_\-good.c}}
\subsubsection[slowheartbeat]{\setlength{\rightskip}{0pt plus 5cm}void slowheartbeat (void)\hspace{0.3cm}{\tt  \mbox{[}static\mbox{]}}}}
\label{mainloop__good_8c_a33}


print \char`\"{}slowheartbeat\char`\"{} message 



Definition at line 344 of file mainloop\_\-good.c.

References message(), and MSG\_\-SLOWHEARTBEAT.

Referenced by main().



\footnotesize\begin{verbatim}345 {
346     message(MSG_SLOWHEARTBEAT);
347 }
\end{verbatim}\normalsize 
\hypertarget{mainloop__good_8c_a46}{
\index{mainloop_good.c@{mainloop\_\-good.c}!writechargen@{writechargen}}
\index{writechargen@{writechargen}!mainloop_good.c@{mainloop\_\-good.c}}
\subsubsection[writechargen]{\setlength{\rightskip}{0pt plus 5cm}void writechargen (int {\em i})\hspace{0.3cm}{\tt  \mbox{[}static\mbox{]}}}}
\label{mainloop__good_8c_a46}


write data to an chargen client 

\begin{Desc}
\item[Parameters:]
\begin{description}
\item[{\em i}]index into pollfds/clients array for chargen session\end{description}
\end{Desc}
If the buffer is not empty, tries to do one write to the filedescriptor associated with the chargen client. 

Definition at line 706 of file mainloop\_\-good.c.

References client\_\-t::chargen, chargen\_\-buf, clients, closeclient(), chargen\_\-client\_\-t::i, message(), and MSG\_\-WRITE.

Referenced by acceptchargen(), and main().



\footnotesize\begin{verbatim}707 {
708     ssize_t         nwrite;
709 
710     nwrite =
711         write(clients[i].fd, chargen_buf + clients[i].chargen.i,
712               sizeof(chargen_buf) - clients[i].chargen.i);
713 
714     switch (nwrite) {
715     case -1:
716         if (errno == EPIPE) {
717             closeclient(i);
718             return;
719         } else if ((errno != EINTR) && (errno != EWOULDBLOCK)) {
720             perror("write()");
721             exit(EXIT_FAILURE);
722         }
723         break;
724     case 0:
725         break;
726     default:
727         message(MSG_WRITE);
728         clients[i].chargen.i += nwrite;
729         clients[i].chargen.i %= sizeof(chargen_buf);
730     }
731 }
\end{verbatim}\normalsize 
\hypertarget{mainloop__good_8c_a43}{
\index{mainloop_good.c@{mainloop\_\-good.c}!writeecho@{writeecho}}
\index{writeecho@{writeecho}!mainloop_good.c@{mainloop\_\-good.c}}
\subsubsection[writeecho]{\setlength{\rightskip}{0pt plus 5cm}void writeecho (int {\em i})\hspace{0.3cm}{\tt  \mbox{[}static\mbox{]}}}}
\label{mainloop__good_8c_a43}


write data to an echo client 

\begin{Desc}
\item[Parameters:]
\begin{description}
\item[{\em i}]index into pollfds/clients array for echo session\end{description}
\end{Desc}
If the buffer is not empty, tries to do one write to the filedescriptor associated with the echo client. 

Definition at line 598 of file mainloop\_\-good.c.

References BUFSIZE, clients, closeclient(), client\_\-t::echo, flowecho(), message(), MSG\_\-EMPTY, MSG\_\-WRITE, echo\_\-client\_\-t::n, echo\_\-client\_\-t::r, and echo\_\-client\_\-t::w.

Referenced by acceptecho(), and main().



\footnotesize\begin{verbatim}599 {
600     ssize_t         nwrite;
601 
602     if (clients[i].echo.n == 0) {
603         message(MSG_EMPTY);
604         flowecho(i);
605         return;
606     }
607     if (clients[i].echo.r > clients[i].echo.w) {
608         nwrite =
609             write(clients[i].fd, clients[i].echo.buf + clients[i].echo.w,
610                   clients[i].echo.r - clients[i].echo.w);
611     } else {
612         nwrite =
613             write(clients[i].fd, clients[i].echo.buf + clients[i].echo.w,
614                   BUFSIZE - clients[i].echo.w);
615     }
616 
617     switch (nwrite) {
618     case -1:
619         if (errno == EPIPE) {
620             closeclient(i);
621         } else if ((errno != EINTR) && (errno != EWOULDBLOCK)) {
622             perror("write()");
623             exit(EXIT_FAILURE);
624         }
625         break;
626     case 0:
627         break;
628     default:
629         message(MSG_WRITE);
630         clients[i].echo.n -= nwrite;
631         clients[i].echo.w += nwrite;
632         clients[i].echo.w %= BUFSIZE;
633         flowecho(i);
634     }
635 }
\end{verbatim}\normalsize 


\subsection{Variable Documentation}
\hypertarget{mainloop__good_8c_a25}{
\index{mainloop_good.c@{mainloop\_\-good.c}!alarms@{alarms}}
\index{alarms@{alarms}!mainloop_good.c@{mainloop\_\-good.c}}
\subsubsection[alarms]{\setlength{\rightskip}{0pt plus 5cm}\hyperlink{structalarm__t}{alarm\_\-t} \hyperlink{mainloop__good_8c_a25}{alarms}\mbox{[}MAXALARMS\mbox{]}\hspace{0.3cm}{\tt  \mbox{[}static\mbox{]}}}}
\label{mainloop__good_8c_a25}


alarms 



Definition at line 134 of file mainloop\_\-good.c.

Referenced by addalarm(), alarmhandler(), checkalarms(), and initalarms().\hypertarget{mainloop__good_8c_a19}{
\index{mainloop_good.c@{mainloop\_\-good.c}!chargen_buf@{chargen\_\-buf}}
\index{chargen_buf@{chargen\_\-buf}!mainloop_good.c@{mainloop\_\-good.c}}
\subsubsection[chargen\_\-buf]{\setlength{\rightskip}{0pt plus 5cm}char \hyperlink{mainloop__good_8c_a19}{chargen\_\-buf}\mbox{[}$\,$\mbox{]} = \char`\"{}0123456789abcdefghijklmnopqrstuv\char`\"{}\hspace{0.3cm}{\tt  \mbox{[}static\mbox{]}}}}
\label{mainloop__good_8c_a19}


characters to return in chargen service 



Definition at line 60 of file mainloop\_\-good.c.

Referenced by writechargen().\hypertarget{mainloop__good_8c_a23}{
\index{mainloop_good.c@{mainloop\_\-good.c}!clients@{clients}}
\index{clients@{clients}!mainloop_good.c@{mainloop\_\-good.c}}
\subsubsection[clients]{\setlength{\rightskip}{0pt plus 5cm}\hyperlink{structclient__t}{client\_\-t} \hyperlink{mainloop__good_8c_a23}{clients}\mbox{[}MAXCLIENTS\mbox{]}\hspace{0.3cm}{\tt  \mbox{[}static\mbox{]}}}}
\label{mainloop__good_8c_a23}


array of client states 



Definition at line 132 of file mainloop\_\-good.c.

Referenced by acceptchargen(), acceptecho(), addclient(), closeclient(), delclient(), flowecho(), initclients(), mainloop(), readecho(), writechargen(), and writeecho().\hypertarget{mainloop__good_8c_a22}{
\index{mainloop_good.c@{mainloop\_\-good.c}!npollfd@{npollfd}}
\index{npollfd@{npollfd}!mainloop_good.c@{mainloop\_\-good.c}}
\subsubsection[npollfd]{\setlength{\rightskip}{0pt plus 5cm}unsigned long \hyperlink{mainloop__good_8c_a22}{npollfd}\hspace{0.3cm}{\tt  \mbox{[}static\mbox{]}}}}
\label{mainloop__good_8c_a22}


number of file descriptors to check 

this is not as bad as it sounds because there's a static array for the struct pollfd anyway (pollfds). using an array for the client states allows using the same index for pollfds and clients.

in a real application, the buffer would be bigger, and therefore wouldn't be placed into the array ;) for demonstration purposes, the buffer is small to make it more interesting 

Definition at line 131 of file mainloop\_\-good.c.

Referenced by addclient(), delclient(), initclients(), and mainloop().\hypertarget{mainloop__good_8c_a24}{
\index{mainloop_good.c@{mainloop\_\-good.c}!pollfds@{pollfds}}
\index{pollfds@{pollfds}!mainloop_good.c@{mainloop\_\-good.c}}
\subsubsection[pollfds]{\setlength{\rightskip}{0pt plus 5cm}struct pollfd \hyperlink{mainloop__good_8c_a24}{pollfds}\mbox{[}MAXCLIENTS\mbox{]}\hspace{0.3cm}{\tt  \mbox{[}static\mbox{]}}}}
\label{mainloop__good_8c_a24}


pollfds for poll() 



Definition at line 133 of file mainloop\_\-good.c.

Referenced by addclient(), delclient(), flowecho(), and mainloop().